% ==============================================================================
% Advanced Computational Geometry for Olympiad & Competitive Programming
% LaTeX Preamble Documentation and Configuration
% ==============================================================================

% ---------------- CORE PACKAGES ----------------
% Font and Encoding
\usepackage[utf8]{inputenc}
\usepackage[T1]{fontenc}
\usepackage{lmodern}
\usepackage{textcomp}
\usepackage{lipsum}

% Page Layout
\usepackage[a4paper, margin=2.5cm]{geometry}
\usepackage{needspace}

% Mathematics
\usepackage{amsmath}
\usepackage{amssymb}
\usepackage{amsthm}
\usepackage{bm}
\usepackage{mathtools}

% Graphics and Color
\usepackage{graphicx}
\usepackage{tikz}
\usepackage{subcaption}
\usetikzlibrary{shapes,arrows,positioning,calc,patterns,decorations.pathreplacing,arrows.meta}

\usepackage{xcolor}
\usepackage{float}
\usepackage{caption}

% Tables
\usepackage{array}
\usepackage{booktabs}
\usepackage{longtable}
\usepackage{multirow}
\usepackage{tabularx}

% Code Listings
\usepackage{listings}

% Algorithms
\usepackage[linesnumbered,ruled,vlined]{algorithm2e}
\SetKwInput{KwInput}{Input}
\SetKwInput{KwOutput}{Output}
\SetKwComment{Comment}{$\triangleright$\ }{}
\DontPrintSemicolon

% Box Environments
\usepackage{mdframed}
\usepackage{tcolorbox}
\usepackage{enumitem}

% Cross-referencing
\usepackage{hyperref}
\usepackage{cleveref}

% Utilities
\usepackage{import}
\usepackage{tocbibind}

% Inconsolata font
\usepackage{inconsolata}

% ---------------- COLOR DEFINITIONS ----------------
% Code syntax highlighting colors
\definecolor{codegreen}{rgb}{0,0.6,0}
\definecolor{codegray}{rgb}{0.5,0.5,0.5}
\definecolor{codepurple}{rgb}{0.58,0,0.82}
\definecolor{backcolour}{rgb}{0.95,0.95,0.92}
\definecolor{magenta}{rgb}{0.58,0,0.82}


% Environment box colors
\definecolor{tipcolor}{RGB}{0,100,0}
\definecolor{warningcolor}{RGB}{139,0,0}
\definecolor{codecolor}{RGB}{0,0,139}
\definecolor{patterncolor}{RGB}{75,0,130}
\definecolor{problemcolor}{RGB}{128,0,128}

% ---------------- HYPERREF CONFIGURATION ----------------
\hypersetup{
    colorlinks=true,
    linkcolor=blue,
    filecolor=magenta,      
    urlcolor=cyan,
    pdftitle={Advanced Computational Geometry for Olympiad \& Competitive Programming},
    pdfpagemode=UseOutlines,
    bookmarksopen=true,
    bookmarksnumbered=true
}

% ---------------- CLEVEREF CONFIGURATION ----------------
\Crefname{problemexample}{Problem Example}{Problem Examples}
\Crefname{gotcha}{Gotcha}{Gotchas}
\Crefname{furtherreading}{Further Reading Item}{Further Reading Items}
\Crefname{openquestion}{Open Question/Trick}{Open Questions/Tricks}
\Crefname{definition}{Definition}{Definitions}
\Crefname{theorem}{Theorem}{Theorems}
\Crefname{lemma}{Lemma}{Lemmas}
\Crefname{corollary}{Corollary}{Corollaries}
\Crefname{proposition}{Proposition}{Propositions}
\Crefname{lstlisting}{Listing}{Listings}
\Crefname{algorithm}{Algorithm}{Algorithms}

% ---------------- NUMBERING CONFIGURATION ----------------
\setcounter{tocdepth}{3}      % Include subsubsections in ToC
\setcounter{secnumdepth}{3}   % Number subsubsections

% ==============================================================================
% CUSTOM ENVIRONMENTS AND COMMANDS DOCUMENTATION
% ==============================================================================

% ---------------- CODE ENVIRONMENTS ----------------
% Usage: For C++ code snippets throughout the book
% \lstset{style=cppstyle}

\lstdefinestyle{cppstyle}{
    language=C++,
    backgroundcolor=\color{backcolour},
    commentstyle=\color{codegreen},
    keywordstyle=\color{magenta},
    numberstyle=\tiny\color{codegray},
    stringstyle=\color{codepurple},
    basicstyle=\ttfamily\scriptsize,
    aboveskip=0.5em,
    belowskip=0.5em,
    lineskip=-0.5pt,
    breakatwhitespace=false,
    breaklines=true,
    keepspaces=true,
    numbers=left,
    numbersep=4pt,
    showspaces=false,
    showstringspaces=false,
    showtabs=false,
    tabsize=4,
    xleftmargin=0.5em,
    framexleftmargin=0em,
    framexrightmargin=0em,
    framextopmargin=0em,
    framexbottommargin=0em,
}

% Set the default style for all listings to cppstyle
\lstset{style=cppstyle}

% Environment for C++ code blocks
% Usage: \begin{cppcode} ... \end{cppcode}
% \newenvironment{cppcode}
% {\begin{lstlisting}[language=C++]}
% {\end{lstlisting}{et{#1}
% }[1][]{%
% ---------------- THEOREM-LIKE ENVIRONMENTS ----------------
% All theorem-like environments are numbered per section

% Mathematical theorems and proofs
\theoremstyle{plain}
\newtheorem{theorem}{Theorem}[section]
\newtheorem{lemma}[theorem]{Lemma}
\newtheorem{corollary}[theorem]{Corollary}
\newtheorem{proposition}[theorem]{Proposition}

% Definitions and examples
\theoremstyle{definition}
\newtheorem{definition}{Definition}[section]
\newtheorem{problemexample}{Problem Example}[section]

% Remarks and notes
\theoremstyle{remark}
\newtheorem*{remark}{Remark}        % Unnumbered
\newtheorem*{note}{Note}            % Unnumbered

% ---------------- EDUCATIONAL BOX ENVIRONMENTS ----------------

% Chapter introduction box
% Usage: At the beginning of each chapter to set context
\newenvironment{chapterintro}{
    \begin{tcolorbox}[
        colback=blue!2,
        colframe=blue!30,
        boxrule=0pt,
        sharp corners,
        before skip=1em,
        after skip=2em
    ]
}{\end{tcolorbox}}

% Chapter roadmap
% Usage: After introduction to show chapter structure
\newenvironment{roadmap}{
    \begin{center}
    \textbf{Chapter Roadmap}\\[1em]
}{\end{center}}

% Intuition box
% Usage: To provide intuitive explanations of complex concepts
\newenvironment{intuition}{
    \begin{mdframed}[
        backgroundcolor=blue!5,
        linewidth=1pt,
        linecolor=blue!50,
        innertopmargin=5pt,
        innerbottommargin=5pt,
        innerrightmargin=7pt,
        innerleftmargin=7pt
    ]
    \textbf{Intuition:} 
}{\end{mdframed}}

% Visual example environment
% Usage: For figures and diagrams
\newenvironment{visualexample}{
    \begin{figure}[H]
    \centering
}{\end{figure}}

% Complexity analysis box
% Usage: To discuss time/space complexity of algorithms
\newenvironment{complexity}{
    \begin{tcolorbox}[
        colback=gray!5,
        colframe=gray!50,
        title={\textbf{Complexity Analysis}},
        coltitle=black,
        fonttitle=\bfseries
    ]
}{\end{tcolorbox}}

% Implementation notes box
% Usage: For implementation-specific details
\newenvironment{implementation}{
    \begin{mdframed}[
        backgroundcolor=green!5,
        linecolor=green!50,
        linewidth=1pt
    ]
    \textbf{Implementation Notes:}
}{\end{mdframed}}

% Tips box
% Usage: For helpful tips and best practices
\newenvironment{tipsbox}{
    \begin{tcolorbox}[
        colback=green!5,
        colframe=green!40,
        title={\textbf{Tips}},
        coltitle=black,
        fonttitle=\bfseries
    ]
}{\end{tcolorbox}}

% Warning box
% Usage: For important warnings and common mistakes
\newenvironment{warning}[1][]{
    \begin{tcolorbox}[
        colback=red!5,
        colframe=red!40,
        title={Warning\ifx&#1&\else: #1\fi},
        coltitle=black,
        fonttitle=\bfseries
    ]
}{\end{tcolorbox}}

% Insight box
% Usage: For key insights and "aha!" moments
\newenvironment{insight}{
    \begin{tcolorbox}[
        colback=blue!5,
        colframe=blue!40,
        title={\textbf{Insight}},
        coltitle=black,
        fonttitle=\bfseries
    ]
}{\end{tcolorbox}}

% Mathematical insight box
% Usage: For mathematical insights specifically
\newenvironment{mathinsight}{
    \Needspace{6\baselineskip}
    \begin{mdframed}[
        backgroundcolor=purple!5,
        linecolor=purple!50,
        linewidth=1pt
    ]
    \textbf{Mathematical Insight:} 
}{\end{mdframed}}

% Debug checklist
% Usage: For debugging guidelines and checklists
\newenvironment{debugchecklist}[1][]{
    \begin{tcolorbox}[
        colback=yellow!5,
        colframe=yellow!50,
        title={\textbf{Debug Checklist\ifx&#1&\else: #1\fi}},
        coltitle=black,
        fonttitle=\bfseries
    ]
    \begin{itemize}[leftmargin=*]
}{\end{itemize}\end{tcolorbox}}

% ---------------- PROBLEM-RELATED ENVIRONMENTS ----------------

% Problem pattern box
% Usage: To describe common problem patterns
\newenvironment{pattern}[1][]{
    \begin{tcolorbox}[
        colback=patterncolor!5,
        colframe=patterncolor!40,
        fonttitle=\bfseries,
        title={#1}
    ]
    % No itemize here; user can write anything inside
}{
    \end{tcolorbox}
}

% Individual problem environment
% Usage: For detailed problem discussions
\newenvironment{problemenv}[1][]{
    \begin{tcolorbox}[
        colback=problemcolor!5,
        colframe=problemcolor!40,
        title={#1},
        coltitle=black,
        fonttitle=\bfseries
    ]
    \begin{itemize}[leftmargin=*,label={}]
}{\end{itemize}\end{tcolorbox}}

% Problem set container
% Usage: To contain a list of practice problems
\newenvironment{problemset}{
    \begin{center}
}{\end{center}}

% Problem list
% Usage: Inside problemset for listing problems
\newenvironment{problemlist}{
    \begin{enumerate}[leftmargin=*]
}{\end{enumerate}}

% Problem item command
% Usage: \problem{Difficulty}{Platform}{Problem Name}{Key Concepts}
\newcommand{\problem}[4]{
    \item \textbf{#1} - \textit{#2} - #3 \hfill \textcolor{gray}{[#4]}
}

% ---------------- GOTCHA ENVIRONMENT ----------------
% Usage: For common pitfalls and mistakes
\newtheorem{gotcha}{Gotcha}[section]

% ---------------- REFERENCE AND ADVANCED CONTENT ----------------

% Summary card
% Usage: Quick reference cards at end of chapters
\newenvironment{summarycard}{
    \begin{center}
}{\end{center}}

% Advanced topics box
% Usage: For advanced or optional material
\newenvironment{advanced}{
    \begin{tcolorbox}[
        colback=gray!5,
        colframe=gray!60,
        title={\textbf{Advanced Topics}},
        coltitle=black,
        fonttitle=\bfseries
    ]
}{\end{tcolorbox}}

% Further reading environment
\newtheorem{furtherreading}{Further Reading}[section]

% Open questions environment
\newtheorem{openquestion}{Open Question/Trick}[section]

% ---------------- EXERCISES ----------------

% Exercises environment
% Usage: For end-of-chapter exercises
\newenvironment{exercises}{
    \begin{enumerate}[leftmargin=*]
}{\end{enumerate}}

% Exercise command
% Usage: \exercise{Exercise description}
\newcommand{\exercise}[1]{\item #1}

% ---------------- UTILITY COMMANDS ----------------

% Cross-linking command
% Usage: \crosslink{\Cref{ch:convexhull}}
\newcommand{\crosslink}[1]{\textit{(see #1)}}

% LaTeX code display
% Usage: \latexcode{section}
\newcommand{\latexcode}[1]{\texttt{\textbackslash #1}}

% Placeholder for formulas
% Usage: \placeholderformula{Area = 1/2 * base * height}
\newcommand{\placeholderformula}[1]{\texttt{#1}}

% Pseudocode placeholder
% Usage: \pseudocode{Sort points by x-coordinate}
\newcommand{\pseudocode}[1]{\texttt{#1}}

% ==============================================================================
% USAGE GUIDE
% ==============================================================================
% 
% CHAPTER STRUCTURE:
% 1. \begin{chapterintro} - Opening hook and motivation
% 2. \begin{roadmap} - Visual overview with TikZ diagram
% 3. Main content sections using appropriate environments
% 4. \begin{summarycard} - Quick reference
% 5. \begin{exercises} - End of chapter problems
%
% SECTION STRUCTURE:
% - \begin{definition} for formal definitions
% - \begin{theorem}, \begin{lemma} for mathematical results
% - \begin{intuition} for intuitive explanations
% - \begin{visualexample} for diagrams
% - \begin{algorithm} for pseudocode
% - \begin{cppcode} for C++ implementations
% - \begin{complexity} for complexity analysis
% - \begin{implementation}, \begin{tipsbox} for practical advice
% - \begin{warning} for common mistakes
% - \begin{gotcha} for tricky edge cases
%
% PROBLEM SECTIONS:
% - \begin{pattern} for problem patterns
% - \begin{problemenv}[Problem Name] for individual problems
% - \begin{problemset} with \begin{problemlist} for practice sets
% - Use \problem{Difficulty}{Platform}{Name}{Concepts} within problemlist
%
% ADVANCED CONTENT:
% - \begin{advanced} for optional advanced material
% - \begin{furtherreading} for references
% - \begin{openquestion} for research topics
%
% CROSS-REFERENCES:
% - Use \label{} and \Cref{} for smart references
% - Use \crosslink{\Cref{}} for explicit cross-references
% ==============================================================================