% frontmatter/introduction.tex
\chapter*{Introduction}
\label{chap:introduction}

Computational geometry is a cornerstone of algorithm design, frequently appearing in programming olympiads and contests. Problems in this domain often test not only algorithmic knowledge but also the ability to implement these algorithms correctly and robustly, especially when dealing with floating-point arithmetic and degenerate cases.

This book is structured into six main parts:
\begin{itemize}
    \item \textbf{Part I: Foundations \& Primitives} lays the groundwork with essential geometric objects, operations, and numerical considerations.
    \item \textbf{Part II: Polygons \& Basic Structures} delves into polygons, their properties, and lattice geometry.
    \item \textbf{Part III: Core Geometric Algorithms} covers fundamental algorithms for intersection, distance, and convex hulls.
    \item \textbf{Part IV: Optimisation Techniques with Geometric Flavor} explores powerful DP optimization techniques like CHT, Slope Trick, and geometry-aware DP.
    \item \textbf{Part V: Advanced Algorithms \& Data Structures} introduces sweep-line algorithms, spatial data structures, parametric search, and other advanced tools.
    \item \textbf{Part VI: Implementation \& Reference} provides practical templates, boilerplate code, and debugging tips.
\end{itemize}

Each chapter typically follows this structure:
\begin{itemize}
    \item \textbf{Formal Theory}: Definitions, theorems, and mathematical underpinnings.
    \item \textbf{Canonical Algorithms}: Pseudocode, explanations, and complexity analysis.
    \item \textbf{Precision \& Implementation Gotchas}: Common errors, numerical stability, and handling degeneracies.
    \item \textbf{Classic and Unusual Use-Cases / Problem Patterns}: Example problems with links and rationale.
    \item \textbf{Template-Quality Code Snippets}: C++17 implementations.
    \item \textbf{Further Reading}: Pointers to authoritative texts and online resources.
    \item \textbf{Open Research Questions or Lesser-Known Tricks}: For deeper exploration.
\end{itemize}

We hope this book serves as a valuable resource for your competitive programming journey.

\textit{Placeholder for more introduction content, how to use the book, target audience, etc.}

\cleardoublepage