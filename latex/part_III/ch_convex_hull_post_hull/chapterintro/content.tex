\begin{chapterintro}
\label{ch:convex-hull-post-hull}
Imagine you're designing a game where characters navigate a complex level filled with obstacles. You need to determine the "visible" area for each character, taking into account the walls and other obstructions. A fundamental tool in solving this and similar problems is the convex hull.

In this chapter, we delve into the fascinating world of convex hulls. They're much more than a geometric curiosity; they serve as essential building blocks for solving a wide variety of computational geometry problems that appear regularly in competitive programming. We will start by exploring the definition and properties of convex hulls, focusing on how to efficiently compute them.

Why does this matter? Consider problems like finding the shortest path that avoids obstacles or identifying the furthest pair of points in a set. Convex hulls provide elegant and efficient solutions to these types of challenges. The key lies in recognizing how the convex hull simplifies a complex problem by reducing the amount of data we need to consider.

\begin{itemize}
    \item \textbf{Real-World Connection:} Imagine a sensor network deployed to monitor an area. The convex hull of the sensors represents the smallest region they cover. If a target moves outside this region, it means the sensors have "lost" it.
\end{itemize}

\begin{itemize}
    \item \textbf{Challenge Problem:} Given a set of points representing the locations of buildings in a city, design an algorithm to find the smallest rectangular plot of land that can enclose *all* the buildings, oriented in any direction.  We will solve this using a clever combination of convex hulls and the rotating calipers technique! This problem combines the concepts of a convex hull with optimization, providing a challenging application of the techniques we will cover.
\end{itemize}

By the end of this chapter, you'll have a solid understanding of what a convex hull is, how to compute it efficiently, and, most importantly, how to apply it to solve a wide range of geometric problems. Get ready to unleash the power of convex hulls!

\end{chapterintro} 