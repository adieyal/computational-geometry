\begin{definition}[Polygon Diameter]
The diameter of a convex polygon is the maximum distance between any two points on the polygon. For a convex polygon, this maximum always occurs between two vertices.
\end{definition}

\begin{algorithm}[H]
\SetAlgoLined
\caption{Finding Polygon Diameter Using Rotating Calipers}
\KwIn{A convex polygon $P$ with vertices in counterclockwise order}
\KwOut{The squared diameter of $P$}

Find the indices $i_{min}$ and $i_{max}$ of the leftmost and rightmost vertices\;
Initialize $j = i_{max}$\;
Initialize $maxDist^2 = 0$\;

\For{each vertex $i$ from $i_{min}$ to $i_{min}+n$}{
    \While{area of triangle formed by vertices $i$, $i+1$, and $j+1$ > area of triangle formed by vertices $i$, $i+1$, and $j$}{
        $j = (j + 1) \bmod n$\;
    }
    Update $maxDist^2 = \max(maxDist^2, dist^2(vertices[i], vertices[j]))$\;
    Update $maxDist^2 = \max(maxDist^2, dist^2(vertices[i+1], vertices[j]))$\;
}
\Return{$maxDist^2$}\;
\end{algorithm}

\begin{insight}
The key insight of this algorithm is that the diameter of a convex polygon must occur between two vertices that form an antipodal pair—vertices that admit parallel supporting lines. The rotating calipers technique efficiently finds all such pairs in linear time.
\end{insight}