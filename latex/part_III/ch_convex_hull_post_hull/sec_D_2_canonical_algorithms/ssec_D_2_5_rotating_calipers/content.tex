\clearpage
\subsection{Rotating Calipers Technique \& Applications}
\label{ssec:rotating_calipers}

\lstinputlisting[language=C++, caption=Rotating Calipers]{code.cpp}

\begin{complexity}
\label{comp:rotating_calipers}
$O(n)$ time.
\end{complexity}
Invariant: j advances monotonically around the polygon

\begin{insight}
The rotating calipers technique is remarkably versatile. Beyond finding the diameter and width of a convex polygon, it can be used to:
\begin{itemize}
    \item Find all antipodal pairs of vertices in linear time
    \item Compute the minimum-area enclosing rectangle
    \item Find the maximum-area/perimeter inscribed triangle/quadrilateral
    \item Determine the minimum-area/perimeter enclosing triangle
    \item Compute the minimum-width annulus containing all points
\end{itemize}
All these problems can be solved in $O(n)$ time after computing the convex hull, making rotating calipers one of the most powerful techniques in computational geometry.
\end{insight}

\subimport{sssec_D_2_5_1_finding_polygon_diameter/}{content}
\subimport{sssec_D_2_5_2_finding_polygon_width/}{content}
\subimport{sssec_D_2_5_3_largest_area_perimeter_triangle/}{content}
\subimport{sssec_D_2_5_4_convex_polygon_intersection/}{content}
\subimport{sssec_D_2_5_5_min_area_perimeter_rectangle/}{content} 