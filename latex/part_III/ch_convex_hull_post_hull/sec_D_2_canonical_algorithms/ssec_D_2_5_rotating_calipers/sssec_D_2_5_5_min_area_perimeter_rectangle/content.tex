Given a convex polygon, find the minimum-area (or minimum-perimeter) rectangle that encloses it.

\begin{insight}
A key insight is that the minimum-area enclosing rectangle must have at least one side flush with an edge of the convex polygon. This allows us to use rotating calipers to consider only $O(n)$ potential rectangles.
\end{insight}

\begin{algorithm}[H]
\SetAlgoLined
\caption{Minimum Area Enclosing Rectangle}
\KwIn{A convex polygon $P$ with vertices in counterclockwise order}
\KwOut{The minimum-area rectangle enclosing $P$}

Initialize four support lines (left, right, top, bottom)\;
Initialize $minArea = \infty$, $bestRect = \emptyset$\;

\For{each edge $e$ of $P$}{
    Update the support lines to be parallel/perpendicular to $e$ and touching $P$\;
    Compute the area of the resulting rectangle\;
    \If{area < minArea}{
        $minArea = area$\;
        $bestRect = $ current rectangle\;
    }
    Determine which support line should advance next\;
    Advance that support line to the next vertex\;
}
\Return{$bestRect$}\;
\end{algorithm}

\begin{visualexample}
Create a sequence of diagrams showing:
1. A convex polygon with a rectangle enclosing it
2. The rectangle rotating to align with different edges of the polygon
3. The support lines advancing during rotation
4. The final minimum-area rectangle
\end{visualexample}

\begin{implementation}
When implementing this algorithm, it's important to:
\begin{itemize}
    \item Correctly determine which support line to advance at each step
    \item Handle parallel edges in the polygon
    \item Account for numerical precision issues when computing areas
\end{itemize}
\end{implementation}

\begin{complexity}
The algorithm runs in $O(n)$ time where $n$ is the number of vertices in the polygon.
\end{complexity}