\clearpage
\subsection{Chan's Algorithm}
\label{ssec:chans_algorithm}

\subimport{./}{algorithm.tex}
\label{algorithm:chans_algorithm}

\begin{insight}
Chan's algorithm is a breakthrough in convex hull computation, achieving an optimal output-sensitive time complexity of $O(n \log h)$, where $n$ is the number of input points and $h$ is the number of points on the convex hull. The core idea is to iteratively "guess" the hull size and combine the strengths of divide-and-conquer and gift-wrapping approaches. By carefully balancing these phases, the algorithm efficiently adapts to the actual output size.
\end{insight}

\begin{mathinsight}
The $O(n \log h)$ complexity arises as follows: For each guess $m \geq h$, the algorithm spends $O(n \log m)$ time computing mini-hulls (using a method like Graham's scan) and $O(m \cdot n/m) = O(n)$ time in the gift-wrapping phase to merge them. Since $m$ doubles each iteration, the total work across all guesses is $O(n \log h)$. This makes Chan's algorithm the first to match the lower bound for output-sensitive convex hull algorithms.
\end{mathinsight}

\lstinputlisting[language=C++, caption=Chan's Algorithm]{code.cpp}

\begin{complexity}
\label{comp:chans_algorithm}
Time complexity: $O(n \log h)$, where $n$ is the number of input points and $h$ is the number of points on the convex hull.
\end{complexity}

\textbf{Note:} This is the optimal output-sensitive algorithm for the convex hull problem.