\begin{figure}[ht]
  \begin{roadmap}
    \label{roadmap:convex_hull_post_hull}
    \tikzset{
      concept/.style={ellipse, draw, fill=blue!10},
      algorithm/.style={rectangle, draw, fill=green!10},
      technique/.style={diamond, draw, fill=orange!10, aspect=2},
      application/.style={rectangle, draw, fill=red!10, rounded corners},
      arrow/.style={->, thick}
    }
    \begin{tikzpicture}[node distance=1.5cm and 3cm]
        % Core concepts
        \node[concept] (convex) {Convex Sets};
        \node[concept, right=of convex] (hull) {Convex Hull};
        \node[concept, below=of hull] (properties) {Properties};
        
        % Algorithms
        \node[algorithm, below=of convex] (graham) {Graham Scan};
        \node[algorithm, below=of graham] (monotone) {Monotone Chain};
        \node[algorithm, right=of monotone] (chan) {Chan's Algorithm};
        \node[algorithm, right=of graham] (dnc) {Divide \& Conquer};
        
        % Post-Hull Techniques
        \node[technique, below=2cm of monotone] (calipers) {Rotating Calipers};
        
        % Applications
        \node[application, below left=of calipers] (diameter) {Diameter};
        \node[application, below=of calipers] (width) {Width};
        \node[application, below right=of calipers] (rectangle) {Min Rectangle};
        
        % Connections
        \draw[arrow] (convex) -- (hull);
        \draw[arrow] (hull) -- (properties);
        \draw[arrow] (hull) -- (graham);
        \draw[arrow] (hull) -- (dnc);
        \draw[arrow] (graham) -- (monotone);
        \draw[arrow] (monotone) -- (chan);
        \draw[arrow] (properties) -- (calipers);
        \draw[arrow] (calipers) -- (diameter);
        \draw[arrow] (calipers) -- (width);
        \draw[arrow] (calipers) -- (rectangle);
    \end{tikzpicture}
  \end{roadmap}
  \caption{From convex hull theory to practical applications}
\end{figure}