\subsection{Convex Set, Convex Combination}
\label{sec:convex-set-convex-combination}

We begin with the core definitions.

\begin{definition}[Convex Set]
\label{def:convex-set}
A set $S$ in the Euclidean plane is called \textbf{convex} if for any two points $p, q \in S$, the entire line segment connecting $p$ and $q$ also lies in $S$.
\end{definition}

\begin{intuition}
Imagine a shape. If you can draw a straight line between any two points inside the shape, and that line stays *completely* within the shape, then the shape is convex. Think of it like this: a convex set has no "dents" or "caves." It's like a perfectly smooth surface.
\end{intuition}

Let's see some examples and non-examples.

\begin{visualexample}
\label{vis:convex-examples}
Draw four diagrams.
1.  A circle, labeled "Convex"
2.  A triangle, labeled "Convex"
3.  A rectangle, labeled "Convex"
4.  A star shape with inward corners, labeled "Non-Convex"
\end{visualexample}

Next, let's define the concept of convex combination.

\begin{definition}[Convex Combination]
A \textbf{convex combination} of points $p_1, p_2, \ldots, p_n$ is a linear combination $\sum_{i=1}^{n} \lambda_i p_i$ where:
\begin{itemize}
    \item Each coefficient $\lambda_i \ge 0$ (non-negative).
    \item The sum of all coefficients equals 1: $\sum_{i=1}^{n} \lambda_i = 1$.
\end{itemize}
\end{definition}

\begin{intuition}
A convex combination represents a weighted average of points. Each point contributes to the 'average' based on its weight ($\lambda_i$). These weights must be non-negative, meaning no point can "cancel out" another. Furthermore, because the weights add up to 1, the combination will always fall *inside* or *on the boundary of* the shape formed by the points.
\end{intuition}

\begin{visualexample}
\label{vis:convex-combination}
Draw a diagram with 3 points, $p_1, p_2, p_3$ that form a triangle.
1.  Show a point that is a convex combination of the three points, and label the point. The diagram should illustrate how it is a weighted average.
2.   Show the same points, $p_1, p_2, p_3$, and a point outside the triangle that is *not* a convex combination of the three points. Highlight the weights of this "combination", and demonstrate why they do not add up to 1.
\end{visualexample}

Here are a couple of crucial cases:

*   If $\lambda_1 = 1$ and all other $\lambda_i = 0$, the convex combination equals $p_1$ (a vertex).
*   If $\lambda_1 = \lambda_2 = 0.5$ and all other $\lambda_i = 0$, the convex combination equals the midpoint of $p_1$ and $p_2$.

A fundamental property ties convex sets and convex combinations together:

\begin{theorem}
A set is convex if and only if it contains all convex combinations of its points.
\label{thm:convex-comb-equivalence}
\end{theorem}

\begin{mathinsight}
This theorem provides an alternative way to define convexity. If we can show that all convex combinations of points within a set also lie within that set, we've proven that the set is convex. This is extremely useful in proofs and helps to solidify the intuition around what "convex" truly means.
\end{mathinsight}
