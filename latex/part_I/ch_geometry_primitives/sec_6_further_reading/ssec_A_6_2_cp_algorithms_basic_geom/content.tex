% part_I/ch_A/sec_A_6_further_reading/ssec_A_6_2_cp_algorithms_basic_geom/content.tex
\subsection{CP-Algorithms: Basic Geometry}
\label{ssec:A.6.2}

\begin{furtherreading}[CP-Algorithms: Basic Geometry (\url{https://cp-algorithms.com/geometry/basic-geometry.html}) \cite{cpalgorithms_basic_geom}]
\label{fr:A.6.2.cpalgo}
This section of CP-Algorithms is a fantastic, concise resource tailored specifically for competitive programmers. It covers many of the same primitives we've discussed, often with direct C++ implementations.
\textbf{Specific Topics Covered Aligning with Our Chapter A}:
\begin{itemize}
    \item \textbf{Point/Vector Structures}: Shows typical C++ structs for points and vectors, including common operations like addition, subtraction, dot/cross products (\Cref{ssec:A.1.1}, \Cref{ssec:A.5.1}).
    \item \textbf{Dot and Cross Product}: Explains their geometric meaning (angle, area, orientation) and formulas (\Cref{sssec:A.1.1.3}, \Cref{sssec:A.1.1.4}).
    \item \textbf{Orientation Test}: Provides logic for the CCW test using cross products (\Cref{ssec:A.1.4}).
    \item \textbf{Distance Formulas}: Covers point-point, point-line, and point-segment distances (\Cref{ssec:A.1.3}).
    \item \textbf{Segment Intersection}: Explains the orientation-based approach to check for segment intersection (\Cref{ssec:A.1.5}).
    \item \textbf{Code Examples}: Provides C++ snippets for many of these, which can be a good reference alongside our template code.
    \item \textbf{Common Pitfalls}: Often discusses issues like floating point precision and integer overflow, reinforcing the lessons from \Cref{sec:A.3}.
\end{itemize}
CP-Algorithms is highly recommended for its practical focus and clear explanations. It's a go-to site for many competitive programmers.
\end{furtherreading}