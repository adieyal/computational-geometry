% part_I/ch_A/sec_A_6_further_reading/ssec_A_6_1_deberg_ch1/content.tex
\subsection{Computational Geometry: Algorithms and Applications by de Berg et al. (Chapter 1)}
\label{ssec:A.6.1}

\begin{furtherreading}[\textit{Computational Geometry: Algorithms and Applications}, 3rd Edition, by M. de Berg, O. Cheong, M. van Kreveld, M. Overmars (Springer, 2008) \cite{deberg2008}]
\label{fr:A.6.1.deberg}
Chapter 1, "Geometric Primitives," of this classic textbook (often just called "de Berg") provides a formal and rigorous introduction to the topics covered in our chapter. It's a standard academic reference.
\textbf{Key Takeaways Relevant to Our Chapter A}:
\begin{itemize}
    \item \textbf{Precise Definitions}: Establishes clear mathematical definitions for points, vectors, lines, segments, and their representations (Sections 1.1, 1.2).
    \item \textbf{Orientation Test}: Discusses the \texttt{Orientation} (or \texttt{CCW}) test in detail, including its derivation from determinants and its importance as a primitive operation (Section 1.3). This aligns with our \Cref{ssec:A.1.4}.
    \item \textbf{Segment Intersection}: Provides a careful treatment of line segment intersection, including handling of degenerate cases (Section 1.3). This corresponds to our \Cref{ssec:A.1.5}.
    \item \textbf{Numerical Issues}: Briefly touches upon the robustness issues with floating-point arithmetic in geometric computations, setting the stage for why exact arithmetic or careful handling is needed.
\end{itemize}
While more theoretical than a competitive programming tutorial, de Berg offers excellent explanations of the "why" behind these primitives and their geometric significance. It's a great place to solidify your understanding if you find the mathematical aspects intriguing. The rest of the book covers many advanced topics we'll encounter later.
\end{furtherreading}