\begin{chapterintro}
\label{intro:geometry_primitives}
Imagine you're designing the AI for a robot navigating a warehouse filled with obstacles. The robot needs to find the shortest path, identify which items are visible from its current location, and even figure out how to grasp objects. Or picture yourself developing a new strategy game where units need to determine lines of sight, optimal firing angles, and whether their formations will collide. These scenarios, common in robotics, game development, and even geographic information systems (GIS), all rely on a fundamental understanding of how to represent and manipulate geometric objects computationally.

This chapter lays the groundwork for your journey into computational geometry. We'll start with the very basics: points, vectors, lines, and how to perform fundamental operations like calculating distances and determining orientations. Mastering these primitives is crucial because they are the building blocks for almost every advanced algorithm you'll encounter later. Without a solid grasp of these foundations, tackling complex problems like finding convex hulls or detecting line segment intersections becomes incredibly challenging.

\textbf{Challenge Problem}: \textit{The Art Gallery Guardian}

An art gallery is shaped like a simple polygon (a polygon that doesn't self-intersect). A single, stationary security camera needs to be placed at one of the polygon's vertices. From its vertex, the camera can see in all directions (360 degrees). Your task is to determine if there exists a vertex from which the entire interior of the gallery is visible. By the end of this chapter, you won't solve this full problem (it's a classic known as the Art Gallery Problem, often needing more advanced techniques), but you'll have the tools to check if a specific point (like a proposed camera location) can see a specific segment (like a wall of the gallery) without obstruction from other walls. This is a key step! More specifically, you'll be able to determine if two points $A$ and $B$ inside or on the boundary of a polygon are mutually visible (i.e., the segment $AB$ does not intersect the interior of any edge of the polygon it's not part of, unless $AB$ itself is an edge).
\end{chapterintro}
