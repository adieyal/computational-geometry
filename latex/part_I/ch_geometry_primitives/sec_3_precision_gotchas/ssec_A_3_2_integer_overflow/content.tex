% part_I/ch_A/sec_A_3_precision_gotchas/ssec_A_3_2_integer_overflow/content.tex
\subsection{Integer Overflow: When Numbers Get Too Big}
\label{ssec:A.3.2}

\begin{intuition}
\label{intuition:A.3.2.int_limits}
Integer types like \texttt{int} and \texttt{long long} have a limited range of values they can store. For a signed 32-bit \texttt{int}, this is roughly $\pm 2 \cdot 10^9$. For a signed 64-bit \texttt{long long}, it's roughly $\pm 9 \cdot 10^{18}$. If a calculation produces a result outside this range, it "wraps around" or overflows, leading to incorrect (and often wildly different) values without any warning from the compiler or runtime!
\end{intuition}

\begin{warning}[Overflow in Geometric Calculations]
\label{warn:A.3.2.geo_overflow}
Geometric calculations, especially those involving products of coordinates, are prime candidates for overflow:
\begin{itemize}
    \item \textbf{Cross Product}: The formula $(x_2-x_1)(y_3-y_1) - (y_2-y_1)(x_3-x_1)$ involves products of differences. If coordinates $x_i, y_i$ are up to $C_{max}$ (e.g., $10^5$), then $x_2-x_1$ can be $2 \cdot C_{max}$. A term like $(x_2-x_1)(y_3-y_1)$ can be up to $(2 C_{max}) \cdot (2 C_{max}) = 4 C_{max}^2$. If $C_{max}=10^5$, this is $4 \cdot (10^5)^2 = 4 \cdot 10^{10}$. This will overflow a 32-bit \texttt{int} (max $\approx 2 \cdot 10^9$) but fits in a 64-bit \texttt{long long} (max $\approx 9 \cdot 10^{18}$). The full cross product expression can be up to $2 \cdot ( (C_{max}-(-C_{max})) \cdot (C_{max}-(-C_{max})) ) = 8 C_{max}^2$ in magnitude, although a more direct $X_A Y_B - X_C Y_D$ with coordinates up to $C_{max}$ results in differences up to $2 C_{max}^2$.
    \item \textbf{Dot Product}: $x_A x_B + y_A y_B$. Similar issue, each term can be $C_{max}^2$. Sum can be $2 C_{max}^2$.
    \item \textbf{Squared Norm/Distance}: $x^2 + y^2$. If $x$ is $C_{max}$, $x^2$ is $C_{max}^2$. Sum can be $2 C_{max}^2$. For example, distance between $(-C_{max}, -C_{max})$ and $(C_{max}, C_{max})$ is $\sqrt{(2C_{max})^2 + (2C_{max})^2}$, squared distance is $8 C_{max}^2$.
\end{itemize}
\end{warning}

\begin{tipsbox}
\label{tips:A.3.2.use_long_long}
\textbf{Default to \texttt{long long} for Coordinates or Calculations}:
\begin{itemize}
    \item If problem constraints allow coordinates up to $10^4 - 10^5$ or more, it's safest to use \texttt{long long} for storing coordinates in your \texttt{Point} struct, or at least for intermediate calculations in cross products, dot products, and squared distances.
    \item Even if coordinates are small enough for \texttt{int} (e.g., $\pm 1000$), $C_{max}^2$ is $10^6$, $2 C_{max}^2$ is $2 \cdot 10^6$, which fits in \texttt{int}. But if $C_{max} \approx 40000$, $C_{max}^2 \approx 1.6 \cdot 10^9$, $2 C_{max}^2 \approx 3.2 \cdot 10^9$, which overflows \texttt{int}. So, a boundary exists around $C_{max} \approx 30000 - 40000$.
    \item If intermediate products are cast to \texttt{long long} before multiplication, e.g., \texttt{(long long)dx1 * dy2}, this helps prevent overflow of the product itself.
\end{itemize}
Using \texttt{long long} consistently for geometric calculations involving products is a good habit in competitive programming. The performance difference is usually negligible.
\end{tipsbox}

\begin{mathinsight}
\label{mathinsight:A.3.2.overflow_check}
Maximum coordinate value $C_{max}$.
\begin{itemize}
    \item A single coordinate difference, e.g., $dx = x_2-x_1$: Range $\approx \pm 2 C_{max}$.
    \item Product of two differences, e.g., $dx \cdot dy$: Range $\approx \pm (2 C_{max})^2 = \pm 4 C_{max}^2$.
    \item Cross product value: Range $\approx \pm 2 \cdot (4 C_{max}^2) = \pm 8 C_{max}^2$ is a loose upper bound; more tightly, it's $\approx \pm 2 C_{max,diff}^2$ where $C_{max,diff}$ is max coordinate difference. If coordinates are $X_i, Y_i$, a product $X_1 Y_2$ is $C_{max}^2$. The difference $X_1 Y_2 - X_2 Y_1$ could be $2 C_{max}^2$.
    \item A \texttt{long long} typically holds up to $\approx 9 \times 10^{18}$. So, $2 C_{max}^2 \le 9 \times 10^{18} \implies C_{max}^2 \le 4.5 \times 10^{18} \implies C_{max} \le \sqrt{4.5 \times 10^{18}} \approx 2.1 \times 10^9$.
    If coordinates themselves can be this large (e.g., problem states $10^9$), then \texttt{long long} is essential for coordinates. Products would need \texttt{\_\_int128} or careful handling. Most contest problems keep coordinates within a range where \texttt{long long} suffices for products.
\end{itemize}
\end{mathinsight}

\begin{debugchecklist}[Integer Overflow]
\label{debug:A.3.2.int_overflow_checklist}
    \item What are the maximum possible coordinate values given by problem constraints?
    \item Am I performing multiplications of coordinates or coordinate differences (e.g., in cross product, dot product, squared distance)?
    \item Are the intermediate products and final results guaranteed to fit within the chosen integer type (\texttt{int} or \texttt{long long})?
    \item Have I explicitly cast operands to \texttt{long long} \emph{before} multiplication if they are stored as \texttt{int}s? E.g., \verb|(long long)val_dx1 * val_dy2|
    \item Test with maximum and minimum coordinate values to check for overflow. E.g., points like $(C_{max}, C_{max}), (C_{max}, -C_{max}), (-C_{max}, -C_{max})$.
\end{debugchecklist}