\subsection{Collinear Points and Degenerate Cases}
\label{ssec:A.3.4}

\begin{intuition}
\label{intuition:A.3.4.degenerate_common}
In geometry, "degenerate" cases are special configurations that might break the general logic of an algorithm. Common examples include:
\begin{itemize}
    \item Three or more points lying on the same line (collinear points).
    \item Multiple points at the exact same location (coincident points).
    \item Segments of zero length (endpoints are coincident).
    \item Polygons with zero area or self-intersections (if they're supposed to be simple).
\end{itemize}
These aren't necessarily "errors" but often require specific handling.
\end{intuition}

\begin{warning}[Algorithms Assume General Position]
\label{warn:A.3.4.general_position}
Many geometric algorithms are first described assuming "general position," meaning no three points are collinear, no two points are coincident, etc. In competitive programming, test cases \textbf{will} include these degenerate configurations. Your code must handle them robustly!
\end{warning}

\begin{insight}
\label{insight:A.3.4.collinearity_is_key}
\textbf{Collinearity is a frequent source of bugs.}
\begin{itemize}
    \item \textbf{Orientation Test}: When \texttt{orientation(P1, P2, P3)} (\Cref{alg:A.2.3.orientation_test}) returns 0, points are collinear. What does this imply for your current algorithm?
        \begin{itemize}
            \item For segment intersection (\Cref{alg:A.2.4.segment_intersect}), it means you need to check for 1D overlap.
            \item For convex hull algorithms (e.g., Graham Scan, \crosslink{\Cref{ssec:D.2.1}}), you need a tie-breaking rule for collinear points (e.g., keep farthest, keep closest, or discard middle ones depending on variant).
            \item For angular sort (\Cref{ssec:A.4.2}), collinear points relative to pivot have the same "angle". Tie-break by distance.
        \end{itemize}
    \item \textbf{Coincident Points}: If multiple input points can be at the same location:
        \begin{itemize}
            \item Does it affect point counts? (e.g., "N distinct points").
            \item Can it create zero-length segments or zero-area triangles/polygons?
            \item Often, pre-processing to remove duplicates or handle them (e.g., by assigning them a count) can simplify later logic.
        \end{itemize}
\end{itemize}
\end{insight}

\begin{tipsbox}
\label{tips:A.3.4.handling_degeneracies}
\textbf{Strategies for Handling Degeneracies}:
\begin{enumerate}
    \item \textbf{Explicit Checks}: Add \texttt{if} statements to handle collinear/coincident cases separately. This is common for segment intersection.
    \item \textbf{Robust Primitives}: Ensure your basic functions like orientation and point-on-segment (\Cref{alg:A.2.4.point_on_segment_collinear}) are exact (use integers) and correctly define behavior for coincident points.
    \item \textbf{Tie-Breaking Rules}: For sorting or selection based on geometric properties (angle, distance), define clear, consistent tie-breaking rules. E.g., if angles are equal, sort by distance. If distances also equal, sort by x-coordinate, then y-coordinate (lexicographical).
    \item \textbf{Symbolic Perturbation}: An advanced technique (rarely needed in contests unless specified) where coordinates are infinitesimally perturbed to remove degeneracies. This is complex to implement correctly.
    \item \textbf{Problem Constraints}: Read carefully! Sometimes problems guarantee "no three points are collinear" or "all points are distinct." If not, assume degeneracies exist.
\end{enumerate}
\end{tipsbox}

\begin{gotcha}
\label{gotcha:A.3.4.zero_length_segments}
\textbf{Zero-Length Segments}: If a segment is defined by $P_1, P_2$ where $P_1=P_2$.
\begin{itemize}
    \item Distance from a point $P_0$ to segment $P_1P_1$ is just distance $P_0P_1$. (Your \Cref{alg:A.2.2.dist_sq_point_segment} should handle this if \texttt{norm\_sq\_AB} is 0.)
    \item Intersection of segment $P_1P_1$ with another segment $P_3P_4$ occurs if point $P_1$ lies on segment $P_3P_4$. (\Cref{alg:A.2.4.segment_intersect} should handle this.)
    \item Cross products involving vector $\vec{P_1P_1}$ (the zero vector) will be zero.
\end{itemize}
Ensure your primitives don't divide by zero or behave unexpectedly when faced with zero-length vectors derived from such segments.
\end{gotcha}

\begin{debugchecklist}[Degenerate Cases]
\label{debug:A.3.4.degenerate_checklist}
    \item What happens if three input points $A, B, C$ are collinear?
        \begin{itemize}
            \item $B$ between $A, C$?
            \item $A$ between $B, C$?
            \item $A, B, C$ are coincident?
            \item $A, B$ coincident, $C$ distinct?
        \end{itemize}
    \item What if an input segment has zero length ($P_1=P_2$)?
    \item What if two input segments are collinear and overlap? Collinear and touch at an endpoint? Collinear and disjoint?
    \item What if an input polygon has collinear vertices? Or repeated vertices?
    \item Test your code with hand-crafted degenerate inputs. For example, for segment intersection:
        \begin{itemize}
            \item \texttt{[(0,0)-(2,2)]} and \texttt{[(1,1)-(3,3)]} (collinear overlap)
            \item \texttt{[(0,0)-(2,2)]} and \texttt{[(2,2)-(3,3)]} (collinear touch at endpoint)
            \item \texttt{[(0,0)-(2,2)]} and \texttt{[(1,0)-(0,1)]} (T-junction, endpoint on segment)
            \item \texttt{[(0,0)-(0,0)]} and \texttt{[(0,0)-(1,1)]} (point segment on segment)
        \end{itemize}
\end{debugchecklist}