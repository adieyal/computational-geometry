% part_I/ch_A/sec_A_1_formal_theory/ssec_A_1_2_lines_segments_rays/content.tex
\subsection{Lines, Segments, and Rays: Paths and Boundaries}
\label{ssec:A.1.2}

Points and vectors are fundamental, but we often care about collections of points forming lines, segments, or rays.

\subsubsection{Representations: Describing Infinite and Finite Paths}
\label{sssec:A.1.2.1}

There are several common ways to represent these linear objects.

\begin{definition}[Line, Segment, Ray Representations]
\label{def:A.1.2.representations}
\begin{itemize}
    \item \textbf{Two Points}: A line can be uniquely defined by two distinct points $P_1, P_2$ that lie on it. A segment is defined by its two endpoints $P_1, P_2$. A ray can be defined by a starting point $P_1$ and another point $P_2$ through which it passes.
    \item \textbf{Point and Direction Vector}: A line can be defined by a point $P_0$ on the line and a non-zero direction vector $\mathbf{d}$ parallel to the line. A ray is defined by its start point $P_0$ and a direction vector $\mathbf{d}$.
    \item \textbf{Implicit Form (for lines)}: A line in 2D can be represented by the equation $ax + by + c = 0$, where $a, b, c$ are constants. The vector $(a,b)$ is a \textbf{normal vector} (perpendicular) to the line. This form is unique up to a scaling factor.
\end{itemize}
\end{definition}

\begin{intuition}
\label{intuition:A.1.2.representations}
Choosing a representation depends on the task:
\begin{itemize}
    \item \textbf{Two Points}: Natural for segments. Often how lines/segments are given in problems.
    \item \textbf{Point and Vector}: Great for parametric forms (\Cref{sssec:A.1.2.2}) and understanding direction.
    \item \textbf{Implicit Form ($ax+by+c=0$)}: Useful for checking if a point lies on a line (plug in coordinates), finding distance from a point to a line, and finding intersection of two lines (solve system of equations). If you have two points $P_1(x_1,y_1), P_2(x_2,y_2)$, you can find $a,b,c$: $a = y_2-y_1$, $b = x_1-x_2$, $c = -(a x_1 + b y_1)$. (Note: this $c$ is $-(ax_1+by_1)$, so $ax_1+by_1+c=0$. Alternatively $c = x_2 y_1 - x_1 y_2$).
\end{itemize}
\end{intuition}

\begin{visualexample}
\label{vis:A.1.2.representations}
% Description of diagram:
% Panel 1: Line L1 shown passing through P1 and P2. Segment S between P1 and P2 highlighted. Ray R1 starting at P1 passing through P2.
% Panel 2: Line L2 shown with point P0 on it and direction vector d parallel to it. Ray R2 starting at P0 in direction d.
% Panel 3: Line L3 shown with equation ax+by+c=0. Normal vector (a,b) shown perpendicular to L3.
Diagram showing:
1. A line $L_1$ passing through points $P_1, P_2$. The segment $P_1P_2$ is highlighted. A ray starting at $P_1$ and going through $P_2$ is also shown.
2. A line $L_2$ defined by point $P_0$ and direction vector $\mathbf{d}$.
3. A line $L_3$ with its equation $ax+by+c=0$. A normal vector $\mathbf{n}=(a,b)$ is shown perpendicular to $L_3$.
\end{visualexample}

\subsubsection{Parametric Form: Tracing the Path}
\label{sssec:A.1.2.2}

\begin{definition}[Parametric Form]
\label{def:A.1.2.parametric}
Given two distinct points $P_1$ and $P_2$, any point $P(t)$ on the line passing through them can be expressed as:
$$ P(t) = P_1 + t(P_2 - P_1) = (1-t)P_1 + tP_2 $$
where $t$ is a real-valued parameter.
\begin{itemize}
    \item \textbf{Line}: $t \in (-\infty, \infty)$.
    \item \textbf{Segment $[P_1, P_2]$}: $t \in [0, 1]$. $P(0)=P_1$, $P(1)=P_2$. Values of $t$ between 0 and 1 give points on the segment.
    \item \textbf{Ray starting at $P_1$ passing through $P_2$}: $t \in [0, \infty)$.
    \item \textbf{Ray starting at $P_2$ passing through $P_1$}: $t \in (-\infty, 1]$ (or equivalently $P_2 + s(P_1-P_2)$ for $s \in [0, \infty)$).
\end{itemize}
If using point $P_0$ and direction vector $\mathbf{d}$: $P(t) = P_0 + t\mathbf{d}$.
For a segment from $P_0$ with length $L$ in direction $\mathbf{d}$ (assuming $\mathbf{d}$ is unit vector): $t \in [0, L]$. (Actually, $P(t) = P_0 + t \cdot \mathbf{d}$ means $t$ scales $\mathbf{d}$. If $\mathbf{d}$ is not unit, $P(1)$ is $P_0+\mathbf{d}$. For segment of length $L$ using unit vector $\mathbf{u}$, $P(t)=P_0+t\mathbf{u}$, $t \in [0,L]$. If $\mathbf{d}$ is $P_2-P_1$, then $t \in [0,1]$ covers segment $P_1P_2$.)
\end{definition}

\begin{intuition}
\label{intuition:A.1.2.parametric}
Think of $t$ as "time".
\begin{itemize}
    \item For a segment $P_1P_2$: At $t=0$, you are at $P_1$. At $t=1$, you are at $P_2$. For $0 < t < 1$, you are somewhere in between. If $t<0$ or $t>1$, you are on the line but outside the segment.
    \item This form is excellent for finding intersection points (solve for $t$) or checking if a point lies on a segment/ray.
\end{itemize}
\end{intuition}

\begin{visualexample}
\label{vis:A.1.2.parametric}
% Description of diagram:
% Points P1 and P2. The infinite line through them is shown.
% The segment P1P2 is highlighted, labeled "t in [0,1]".
% The ray from P1 through P2 is shown, labeled "t in [0,inf)".
% A point for t=-0.5 (on line, outside segment, beyond P1) and t=1.5 (on line, outside segment, beyond P2) are marked.
Line through $P_1, P_2$.
Point $P(0)=P_1$, $P(0.5)$ (midpoint), $P(1)=P_2$.
Also show $P(-0.5)$ and $P(1.5)$ on the extended line.
Labels for segment ($0 \le t \le 1$) and ray ($t \ge 0$) parts.
\end{visualexample}

\subsubsection{Properties: Slope and Intercepts (Mainly for Lines)}
\label{sssec:A.1.2.3}

\begin{definition}[Slope, Intercepts]
\label{def:A.1.2.slope_intercept}
For a non-vertical line:
\begin{itemize}
    \item \textbf{Slope ($m$)}: Measures the steepness. Given two points $(x_1, y_1)$ and $(x_2, y_2)$ on the line with $x_1 \neq x_2$, $m = \frac{y_2 - y_1}{x_2 - x_1}$.
    \item \textbf{Y-intercept ($c$ or $b$)}: The y-coordinate where the line crosses the y-axis. The line equation can be $y = mx + c$.
    \item \textbf{X-intercept}: The x-coordinate where the line crosses the x-axis.
\end{itemize}
For a line $ax+by+c=0$:
\begin{itemize}
    \item If $b \neq 0$, slope $m = -a/b$. Y-intercept is $-c/b$.
    \item If $b = 0$ (so $ax+c=0$, $a \neq 0$): Vertical line $x = -c/a$. Slope is undefined.
    \item If $a = 0$ (so $by+c=0$, $b \neq 0$): Horizontal line $y = -c/b$. Slope is 0.
\end{itemize}
\end{definition}

\begin{intuition}
\label{intuition:A.1.2.slope_intercept}
Slope-intercept form $y=mx+c$ is familiar but has issues with vertical lines (infinite slope). The $ax+by+c=0$ form is more general.
In competitive programming, we often work with vectors or pairs of points, and calculate slope only if needed, being careful about division by zero for vertical lines.
Two lines are parallel if their slopes are equal or both are vertical. They are perpendicular if $m_1 \cdot m_2 = -1$ (unless one is horizontal and other vertical). Using dot product of direction vectors or normal vectors is more robust for checking parallelism/perpendicularity. (Parallel: cross product of direction vectors is 0. Perpendicular: dot product of direction vectors is 0).
\end{intuition}

\begin{gotcha}
\label{gotcha:A.1.2.vertical_lines_slope}
\textbf{Vertical Lines and Slope}: The concept of slope breaks down for vertical lines ($x_1 = x_2$). Always handle this case separately if your algorithm relies on slope calculation. Using vector directions (e.g., $(0, \Delta y)$ for vertical) or the $ax+by+c=0$ form avoids this issue.
\end{gotcha}