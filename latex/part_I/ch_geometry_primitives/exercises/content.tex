\clearpage
\begin{exercises}
\label{ex:A}
\exercise{\label{ex:A.1}Implement a \texttt{Point} struct for \texttt{long long} coordinates with methods for addition, subtraction, dot product, cross product, and squared norm. Test it thoroughly with various inputs, including edge cases like zero vectors and coincident points. (\Cref{ssec:A.5.1})}
\exercise{\label{ex:A.2}Write a function \texttt{orientation(P, Q, R)} that returns -1 (CW), 0 (Collinear), or 1 (CCW) for three points P, Q, R using only integer arithmetic. Ensure it correctly handles potential overflows by using \texttt{long long} for intermediate cross product calculations. (\Cref{def:A.1.4.orientation}, \Cref{alg:A.2.3.orientation_test})}
\exercise{\label{ex:A.3}Implement a function \texttt{segment\_intersect(P1, P2, P3, P4)} that correctly handles general cases, collinear overlaps, and endpoint touching. Test with edge cases like T-junctions, overlapping collinear segments, and segments that are single points. (\Cref{ssec:A.1.5}, \Cref{alg:A.2.4.segment_intersect})}
\exercise{\label{ex:A.4}Given a point \(P_0\) and a list of other points \(P_1, \dots, P_N\), sort these points angularly around \(P_0\). Implement this using both \texttt{atan2} and a cross-product based comparator. Discuss the pros and cons of each method for competitive programming, especially concerning precision, speed, and handling of points in different quadrants relative to \(P_0\). (\Cref{ssec:A.4.2})}
\exercise{\label{ex:A.5}What is the maximum possible value of the $z$-component of the cross product $(P_2-P_1) \times (P_3-P_1)$ if the coordinates of $P_1, P_2, P_3$ are integers between $-10^5$ and $10^5$? What data type is needed to store this result without overflow? (Hint: $(x_2-x_1)$ can be $2 \cdot 10^5$). (\Cref{ssec:A.3.2})}
\exercise{\label{ex:A.6}Describe how you would modify the point-segment distance algorithm (\Cref{alg:A.2.2.dist_point_segment}) to return not just the distance, but also the coordinates of the closest point on the segment to the query point $P_0$.}
\exercise{\label{ex:A.7}Consider three distinct collinear points $A, B, C$. How can you use only dot products (and no \texttt{sqrt} or division if possible) involving vectors formed by these points to determine if $B$ lies strictly between $A$ and $C$? (Hint: consider vectors $\vec{BA}$ and $\vec{BC}$).}
\exercise{\label{ex:A.8}(Challenge) Given a simple polygon (a list of vertices in order) and two points $A$ and $B$ (which can be inside, outside, or on the boundary). Determine if the segment $AB$ intersects any edge of the polygon. What further checks are needed if $A$ and $B$ are inside and you want to know if they are mutually visible (i.e., $AB$ doesn't cross any edge)? Consider cases where $A$ or $B$ are vertices of the polygon, or $AB$ is collinear with an edge.}
\end{exercises}
