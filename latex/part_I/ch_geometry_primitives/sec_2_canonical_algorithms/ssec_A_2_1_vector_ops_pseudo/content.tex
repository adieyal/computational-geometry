% part_I/ch_A/sec_A_2_canonical_algorithms/ssec_A_2_1_vector_ops_pseudo/content.tex
\subsection{Vector Operations in Pseudocode}
\label{ssec:A.2.1}

These algorithms assume points/vectors are represented as structures with \texttt{x} and \texttt{y} members (e.g., \texttt{P.x}, \texttt{P.y}).

\begin{algorithm}[H]
\caption{Dot Product of 2D Vectors $\mathbf{u}, \mathbf{v}$}
\label{alg:A.2.1.dot_product}
\KwInput{Vector $\mathbf{u} = (u_x, u_y)$, Vector $\mathbf{v} = (v_x, v_y)$}
\KwOutput{Scalar dot product $\mathbf{u} \cdot \mathbf{v}$}
\Return{$u_x \cdot v_x + u_y \cdot v_y$}
\end{algorithm}
\begin{complexity}
\label{comp:A.2.1.dot_product}
$O(1)$ time. Two multiplications and one addition.
\end{complexity}

\begin{algorithm}[H]
\caption{Cross Product of 2D Vectors $\mathbf{u}, \mathbf{v}$ (z-component)}
\label{alg:A.2.1.cross_product_2d}
\KwInput{Vector $\mathbf{u} = (u_x, u_y)$, Vector $\mathbf{v} = (v_x, v_y)$}
\KwOutput{Scalar cross product $u_x v_y - u_y v_x$}
\Return{$u_x \cdot v_y - u_y \cdot v_x$}
\end{algorithm}
\begin{complexity}
\label{comp:A.2.1.cross_product_2d}
$O(1)$ time. Two multiplications and one subtraction.
\end{complexity}
\begin{warning}
\label{warn:A.2.1.cross_product_overflow}
Ensure the data type for the return value can hold $2 \cdot C_{max}^2$ if coordinates are up to $C_{max}$ (see \Cref{tips:A.1.4.integer_arithmetic}). Use \texttt{long long} if inputs are \texttt{int}.
\end{warning}

\begin{algorithm}[H]
\caption{Norm Squared of a 2D Vector $\mathbf{v}$}
\label{alg:A.2.1.norm_sq}
\KwInput{Vector $\mathbf{v} = (v_x, v_y)$}
\KwOutput{Scalar norm squared $|\mathbf{v}|^2$}
\Return{$v_x \cdot v_x + v_y \cdot v_y$}
\end{algorithm}
\begin{complexity}
\label{comp:A.2.1.norm_sq}
$O(1)$ time. Two multiplications and one addition.
\end{complexity}
\begin{insight}
\label{insight:A.2.1.norm_sq_vs_norm}
Using \texttt{norm\_sq} avoids computing the square root, which is slower and can introduce floating point errors if the coordinates are integers. This method is ideal for comparing distances.
\end{insight}

\begin{algorithm}[H]
\caption{Rotate 2D Vector $\mathbf{v}$ CCW by Angle $\theta$}
\label{alg:A.2.1.rotate_vector}
\KwInput{Vector $\mathbf{v} = (v_x, v_y)$, Angle $\theta$ (in radians)}
\KwOutput{Rotated vector $\mathbf{v}' = (v'_x, v'_y)$}
$c \leftarrow \cos(\theta)$\;
$s \leftarrow \sin(\theta)$\;
$v'_x \leftarrow v_x \cdot c - v_y \cdot s$\;
$v'_y \leftarrow v_x \cdot s + v_y \cdot c$\;
\Return{$(v'_x, v'_y)$}
\end{algorithm}
\begin{complexity}
\label{comp:A.2.1.rotate_vector}
$O(1)$ time, but involves trigonometric functions which are computationally more expensive than simple arithmetic.
\end{complexity}
\begin{implementation}
\label{impl:A.2.1.rotate_vector_precision}
If $\theta$ is a special angle like $90^\circ$ or $180^\circ$, use direct coordinate manipulation to avoid \texttt{sin}/\texttt{cos} and potential precision loss.
For $90^\circ$ CCW: $v'_x \leftarrow -v_y$; $v'_y \leftarrow v_x$.
For $90^\circ$ CW: $v'_x \leftarrow v_y$; $v'_y \leftarrow -v_x$.
For $180^\circ$: $v'_x \leftarrow -v_x$; $v'_y \leftarrow -v_y$.
\end{implementation}

\begin{algorithm}[H]
\caption{Project Vector $\mathbf{a}$ onto Non-Zero Vector $\mathbf{b}$}
\label{alg:A.2.1.project_vector}
\KwInput{Vector $\mathbf{a}$, Vector $\mathbf{b}$ (non-zero)}
\KwOutput{Vector projection of $\mathbf{a}$ onto $\mathbf{b}$}
\lIf{$\mathbf{b}.x = 0$ \textbf{and} $\mathbf{b}.y = 0$} { \Return{\textbf{Error} or $\mathbf{0}$} \Comment{Cannot project onto zero vector} }
dot\_prod $\leftarrow \text{dot\_product}(\mathbf{a}, \mathbf{b})$\;
norm\_sq\_b $\leftarrow \text{norm\_sq}(\mathbf{b})$\;
\If{norm\_sq\_b is very close to 0 (for floats) or is 0 (for integers)} {
    \Return{\textbf{Error} or $(0,0)$ } \Comment{Should be caught by initial check if exact 0}
}
scalar\_factor $\leftarrow \text{dot\_prod} / \text{norm\_sq\_b}$\;
projected\_vector\_x $\leftarrow \mathbf{b}.x \cdot \text{scalar\_factor}$\;
projected\_vector\_y $\leftarrow \mathbf{b}.y \cdot \text{scalar\_factor}$\;
\Return{$(projected\_vector\_x, projected\_vector\_y)$}
\end{algorithm}
\begin{complexity}
\label{comp:A.2.1.project_vector}
$O(1)$ time. Involves dot product, norm squared, one division, two multiplications.
\end{complexity}
\begin{warning}
\label{warn:A.2.1.project_vector_zero}
Always check if vector $\mathbf{b}$ is the zero vector before dividing by its norm squared to prevent division by zero. If $\mathbf{b}$ is zero, the projection is often considered undefined or the zero vector depending on context.
\end{warning}