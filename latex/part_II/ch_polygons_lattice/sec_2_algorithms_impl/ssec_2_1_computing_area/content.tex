% Content for Computing Area 

\subsection{Computing Polygon Area}

\begin{algorithm}[H]
\caption{Shoelace Algorithm}
\KwInput{$vertices$}
\KwOutput{Area of the polygon}
$n \gets vertices.size()$\;
$area \gets 0$\;
\For{$i \gets 0$ \KwTo $n-1$}{
    $j \gets (i+1) \bmod n$\;
    $area \gets area + vertices[i].x \times vertices[j].y$\;
    $area \gets area - vertices[j].x \times vertices[i].y$\;
}
\Return{$|area| / 2$}
\end{algorithm}

\begin{complexity}
\begin{itemize}
\item \textbf{Time:} $O(n)$ where $n$ is the number of vertices
\item \textbf{Space:} $O(1)$ additional space
\item \textbf{When to use:} Always for simple polygons; works for both convex and non-convex
\end{itemize}
\end{complexity}

\begin{implementation}
\begin{tipsbox}
\begin{itemize}
\item \textbf{Overflow:} For large coordinates, intermediate products can overflow. Use \texttt{long long} or \texttt{\_\_int128}.
\item \textbf{Sign:} The raw result's sign indicates orientation (CW/CCW). Take absolute value for area.
\item \textbf{Precision:} For integer coordinates, the result before division by 2 is always even.
\end{itemize}
\end{tipsbox}
\end{implementation} 