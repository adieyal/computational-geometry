% appendix/app_B_master.tex
\chapter{Problem List (JSON format)}
\label{ch:AppendixB}

% The content will be directly in this file, or in a content.tex if you prefer
% For simplicity, including directly:

\section*{Problem List} % Unnumbered section
\label{sec:AppB.problems} % Label if needed

\begin{verbatim}
[
  {
    "problem_name": "Points and Lines",
    "source_platform": "Codeforces Gym",
    "problem_id": "100187B",
    "problem_url": "https://codeforces.com/gym/100187/problem/B",
    "relevant_chapters_sections": ["ch:A", "ssec:A.1.3", "ssec:A.1.4"],
    "brief_description_of_fit": "Direct application of orientation predicate and point-line distance."
  },
  {
    "problem_name": "Polygon",
    "source_platform": "TopCoder SRM",
    "problem_id": "144 Div1 Easy",
    "problem_url": " (Search for TopCoder SRM 144 problem archive) ",
    "relevant_chapters_sections": ["ch:A", "ssec:A.1.1.4", "ssec:A.4.2"],
    "brief_description_of_fit": "Sorting points by angle using atan2 or cross product comparison."
  },
  % ... more problems in JSON format ...
  {
    "problem_name": "Polygon Area",
    "source_platform": "CSES",
    "problem_id": "2191",
    "problem_url": "https://cses.fi/problemset/task/2191",
    "relevant_chapters_sections": ["ch:B", "ssec:B.2.1"],
    "brief_description_of_fit": "Direct application of Shoelace formula."
  }
]
\end{verbatim}
% Note: For a very long JSON, you might consider \usepackage{verbatimfiles} 
% and putting the JSON in a separate .json file to be included.
% For now, \begin{verbatim} is fine.
