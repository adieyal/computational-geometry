\begin{programmingproblem}{XOR on Segment}{4 seconds}{256 megabytes}{C}{prob:xor_on_segment}
    \problemdescription{
        You are given an array $\sequence{a_1, a_2, \ldots, a_n}$ of $n$ integers. You can perform two types of operations on this array:
        \begin{itemize}
            \item Calculate the sum of elements on the segment $[l, r]$, i.e., compute $a_l + a_{l+1} + \ldots + a_r$.
            \item Apply the xor operation with a given number $x$ to each array element on the segment $[l, r]$, i.e., set $a_i := a_i \oplus x$ for all $l \leq i \leq r$.
        \end{itemize}
        Here, $\oplus$ denotes the bitwise xor operation. This operation exists in all modern programming languages, for example, in C++ and Java it is marked as ``\texttt{\^}'', in Pascal as ``\texttt{xor}''.
        \\
        You are given a list of $m$ operations. For each sum query, print the result.
    }
    
    \probleminput{
        The first line contains integer $n$ ($1 \leq n \leq 10^5$) — the size of the array.\\
        The second line contains $n$ space-separated integers $a_1, a_2, \ldots, a_n$ ($0 \leq a_i \leq 10^6$) — the original array.\\
        The third line contains integer $m$ ($1 \leq m \leq 5 \cdot 10^4$) — the number of operations.\\
        Each of the next $m$ lines describes an operation. The $i$-th line starts with integer $t_i$ ($1 \leq t_i \leq 2$) — the type of the operation.\\
        If $t_i = 1$, then this is a sum query, and next follow two integers $l_i, r_i$ ($1 \leq l_i \leq r_i \leq n$).\\
        If $t_i = 2$, then this is an xor update, and next follow three integers $l_i, r_i, x_i$ ($1 \leq l_i \leq r_i \leq n$, $1 \leq x_i \leq 10^6$).
    }
    
    \problemoutput{
        For each query of type 1, print the sum of numbers on the given segment in a single line. Print the answers in the order in which the queries appear in the input.
    }
    
    \begin{problemexamples}
        \begin{example}{
            5\\
            4 10 3 13 7\\
            8\\
            1 2 4\\
            2 1 3 3\\
            1 2 4\\
            1 3 3\\
            2 2 5 5\\
            1 1 5\\
            2 1 2 10\\
            1 2 3
        }{
            26\\
            22\\
            0\\
            34\\
            11
        }
        \end{example}
        \begin{example}{
            6\\
            4 7 4 0 7 3\\
            5\\
            2 2 3 8\\
            1 1 5\\
            2 3 5 1\\
            2 4 5 6\\
            1 2 3
        }{
            38\\
            28
        }
        \end{example}
    \end{problemexamples}
    
    \problemnotes{
        The bitwise xor operation $\oplus$ is defined as follows: for each bit position, the result is 1 if the bits in the operands are different, and 0 otherwise.\\
        For example, $5 \oplus 3 = 6$ because $5 = 101_2$, $3 = 011_2$, and $101_2 \oplus 011_2 = 110_2 = 6$.
    }
\end{programmingproblem}