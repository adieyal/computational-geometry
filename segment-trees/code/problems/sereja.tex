\begin{programmingproblem}{Sereja and Brackets}{1 second}{256 megabytes}{C}{prob:sereja}
    \problemdescription{
        Sereja has a bracket sequence \sequence{s_1, s_2, \ldots, s_n}, or, in other words, 
        a string \mathvar{s} of length \mathvar{n}, consisting of characters ``('' and ``)''.\\
        Sereja needs to answer \mathvar{m} queries, each of them is described by two integers 
        \mathvar{l_i, r_i} (\mathvar{1 \leq l_i \leq r_i \leq n}). The answer to the \mathvar{i}-th 
        query is the length of the maximum \term{correct bracket subsequence} of sequence 
        \sequence{s_{l_i}, s_{l_i + 1}, \ldots, s_{r_i}}. Help Sereja answer all queries.\\
        You can find the definitions for a \term{subsequence} and a \term{correct bracket sequence} 
        in the notes.
    }
    
    \probleminput{
        The first line contains a sequence of characters \sequence{s_1, s_2, \ldots, s_n} 
        (\mathvar{1 \leq n \leq 10^6}) without any spaces. Each character is either a ``('' or a ``)''. 
        The second line contains integer \mathvar{m} (\mathvar{1 \leq m \leq 10^5}) --- the number of queries. 
        Each of the next \mathvar{m} lines contains a pair of integers. The \mathvar{i}-th line contains 
        integers \mathvar{l_i, r_i} (\mathvar{1 \leq l_i \leq r_i \leq n}) --- the description of the 
        \mathvar{i}-th query.
    }
    
    \problemoutput{
        Print the answer to each question on a single line. Print the answers in the order they go in the input.
    }
    
    \begin{problemexamples}
        \begin{example}{
            ())(())(())(\\
            7\\
            1 1\\
            2 3\\
            1 2\\
            1 12\\
            8 12\\
            5 11\\
            2 10
        }{
            0\\
            0\\
            2\\
            10\\
            4\\
            6\\
            6
        }
        \end{example}
    \end{problemexamples}
    
    % For very long examples, use:
    % \begin{longexample}{very long input here}{very long output here}
    % \end{longexample}
    
    \problemnotes{
        A \term{subsequence} of length \mathvar{|x|} of string \mathvar{s = s_1s_2\ldots s_{|s|}} 
        (where \mathvar{|s|} is the length of string \mathvar{s}) is string 
        \mathvar{x = s_{k_1}s_{k_2}\ldots s_{k_{|x|}}} (\mathvar{1 \leq k_1 < k_2 < \ldots < k_{|x|} \leq |s|}).\\
        A \term{correct bracket sequence} is a bracket sequence that can be transformed into a correct 
        arithmetic expression by inserting characters ``1'' and ``+'' between the characters of the string. 
        For example, bracket sequences ``()()'', ``(())'', are correct (the resulting expressions 
        ``(1)+(1)'', ``((1+1)+1)''), and ``)('' and ``('' are not.\\
        For the third query required sequence will be \sequence{()}.\\
        For the fourth query required sequence will be \sequence{()()()()}.
    }
\end{programmingproblem}

