\begin{programmingproblem}{Lucky Queries}{3 seconds}{256 megabytes}{C}{prob:lucky-queries}
    \problemdescription{
        Petya loves lucky numbers very much. Everybody knows that lucky numbers are positive integers whose decimal record contains only the lucky digits 4 and 7. For example, numbers 47, 744, 4 are lucky and 5, 17, 467 are not.\\
        Petya brought home string \mathvar{s} with the length of \mathvar{n}. The string only consists of lucky digits. The digits are numbered from the left to the right starting with 1. Now Petya should execute \mathvar{m} queries of the following form:
        \begin{itemize}
            \item \texttt{switch l r} — "switch" digits (i.e. replace them with their opposites) at all positions with indexes from \mathvar{l} to \mathvar{r}, inclusive: each digit 4 is replaced with 7 and each digit 7 is replaced with 4 (\mathvar{1 \leq l \leq r \leq n});
            \item \texttt{count} — find and print on the screen the length of the longest non-decreasing subsequence of string \mathvar{s}.
        \end{itemize}
        A subsequence of a string \mathvar{s} is a string that can be obtained from \mathvar{s} by removing zero or more of its elements. A string is called non-decreasing if each successive digit is not less than the previous one.\\
        Help Petya process the requests.
    }
    
    \probleminput{
        The first line contains two integers \mathvar{n} and \mathvar{m} (\mathvar{1 \leq n \leq 10^6}, \mathvar{1 \leq m \leq 3 \cdot 10^5}) — the length of the string \mathvar{s} and the number of queries correspondingly.\\
        The second line contains \mathvar{n} lucky digits without spaces — Petya's initial string.\\
        Next \mathvar{m} lines contain queries in the form described in the statement.
    }
    
    \problemoutput{
        For each query \texttt{count} print an answer on a single line.
    }
    
    \begin{problemexamples}
        \begin{example}{
            2 3\\
            47\\
            count\\
            switch 1 2\\
            count
        }{
            2\\
            1
        }
        \end{example}
        \begin{example}{
            3 5\\
            747\\
            count\\
            switch 1 1\\
            count\\
            switch 1 3\\
            count
        }{
            2\\
            3\\
            2
        }
        \end{example}
    \end{problemexamples}
    
    \problemnotes{
        In the first sample the chronology of string \mathvar{s} after some operations are fulfilled is as follows (the sought maximum subsequence is marked with bold):\\
        47\\
        74\\
        74\\
        In the second sample:\\
        747\\
        447\\
        447\\
        774\\
        774
    }
\end{programmingproblem}
