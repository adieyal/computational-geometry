\begin{programmingproblem}{Till I Collapse}{2 seconds}{256 megabytes}{C}{prob:till_i_collapse}
    \problemdescription{
        Rick and Morty want to find MR. PBH and they can't do it alone. So they need Mr. Meeseeks. They have generated $n$ Mr. Meeseeks, standing in a line numbered from $1$ to $n$. Each of them has his own color. The $i$-th Mr. Meeseeks' color is $a_i$.\\
        Rick and Morty are gathering their army and they want to divide Mr. Meeseeks into some squads. They don't want their squads to be too colorful, so each squad should have Mr. Meeseeks of at most $k$ different colors. Also, each squad should be a continuous subarray of Mr. Meeseeks in the line. Meaning that for each $1 \leq i \leq e \leq j \leq n$, if Mr. Meeseeks number $i$ and Mr. Meeseeks number $j$ are in the same squad then Mr. Meeseeks number $e$ should be in that same squad.\\
        Also, each squad needs its own presidio, and building a presidio needs money, so they want the total number of squads to be minimized.\\
        Rick and Morty haven't finalized the exact value of $k$, so in order to choose it, for each $k$ between $1$ and $n$ (inclusive) they need to know the minimum number of presidios needed.
    }

    \probleminput{
        The first line of input contains a single integer $n$ ($1 \leq n \leq 10^5$) - number of Mr. Meeseeks.\\
        The second line contains $n$ integers $a_1, a_2, \ldots, a_n$ separated by spaces ($1 \leq a_i \leq n$) --- colors of Mr. Meeseeks in the order they are standing in a line.
    }

    \problemoutput{
        In the first and only line of output print $n$ integers separated by spaces. The $i$-th integer should be the minimum number of presidios needed if the value of $k$ is $i$.
    }

    \begin{problemexamples}
        \begin{example}{
            5\\
            1\ 3\ 4\ 3\ 3
        }{
            4\ 2\ 1\ 1\ 1
        }
        \end{example}
        \begin{example}{
            8\\
            1\ 5\ 7\ 8\ 1\ 7\ 6\ 1
        }{
            8\ 4\ 3\ 2\ 1\ 1\ 1\ 1
        }
        \end{example}
    \end{problemexamples}


    \problemnotes{
        For the first sample testcase, some optimal ways of dividing the army into squads for each $k$ are:
    }

    \begin{itemize}
        \item $k=1$: $[1], [3], [4], [3, 3]$
        \item $k=2$: $[1], [3, 4, 3, 3]$
        \item $k=3$: $[1, 3, 4, 3, 3]$
        \item $k=4$: $[1, 3, 4, 3, 3]$
        \item $k=5$: $[1, 3, 4, 3, 3]$
    \end{itemize}

    \problemnotes{
        For the second sample testcase, some optimal ways of dividing the army into squads for each $k$ are:
    }

    \begin{itemize}
        \item $k=1$: $[1], [5], [7], [8], [1], [7], [6], [1]$
        \item $k=2$: $[1, 5], [7, 8], [1, 7], [6, 1]$
        \item $k=3$: $[1, 5, 7], [8], [1, 7, 6, 1]$
        \item $k=4$: $[1, 5, 7, 8], [1, 7, 6, 1]$
        \item $k=5$ to $k=8$: $[1, 5, 7, 8, 1, 7, 6, 1]$
    \end{itemize}
\end{programmingproblem}
