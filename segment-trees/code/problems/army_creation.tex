\begin{programmingproblem}{Army Creation}{2 seconds}{256 megabytes}{C}{prob:armycreation}
    \problemdescription{
        As you might remember from our previous rounds, Vova really likes computer games. Now he is playing a strategy game known as Rage of Empires.\\
        In the game Vova can hire $n$ different warriors; the $i$-th warrior has the type $a_i$. Vova wants to create a balanced army by hiring some subset of warriors. An army is called \term{balanced} if for each type of warrior present in the game there are not more than $k$ warriors of this type in the army. Of course, Vova wants his army to be as large as possible.\\
        To make things more complicated, Vova has to consider $q$ different plans of creating his army. The $i$-th plan allows him to hire only warriors whose numbers are not less than $l_i$ and not greater than $r_i$.\\
        Help Vova to determine the largest size of a balanced army for each plan.\\
        Be aware that the plans are given in a modified way. See input section for details.
    }
    
    \probleminput{
        The first line contains two integers $n$ and $k$ ($1 \leq n, k \leq 10^5$).\\
        The second line contains $n$ integers $a_1, a_2, \ldots, a_n$ ($1 \leq a_i \leq 10^5$).\\
        The third line contains one integer $q$ ($1 \leq q \leq 10^5$) --- the number of plans.\\
        Then $q$ lines follow. The $i$-th line contains two numbers $x_i$ and $y_i$ ($1 \leq x_i, y_i \leq n$) which represent the $i$-th plan.\\
        You have to keep track of the answer to the last plan (let's call it \texttt{last}). In the beginning \texttt{last} $= 0$. Then to restore values of $l_i$ and $r_i$ for the $i$-th plan, you have to do the following:\\
        \[
        l_i = ((x_i + \texttt{last}) \bmod n) + 1;
        \]
        \[
        r_i = ((y_i + \texttt{last}) \bmod n) + 1;
        \]
        If $l_i > r_i$, swap $l_i$ and $r_i$.
    }
    
    \problemoutput{
        Print $q$ numbers. The $i$-th number must be equal to the maximum size of a balanced army when considering the $i$-th plan.
    }
    
    \begin{problemexamples}
        \begin{example}{
            6 2\\
            1 1 1 2 2 2\\
            5\\
            1 6\\
            4 3\\
            1 1\\
            2 6\\
            2 6
        }{
            2\\
            4\\
            1\\
            3\\
            2
        }
        \end{example}
    \end{problemexamples}
    
    \problemnotes{
        In the first example, the real plans are:\\
        1 2\\
        1 6\\
        6 6\\
        2 4\\
        4 6\\
    }
\end{programmingproblem}
