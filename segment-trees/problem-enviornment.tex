% Programming Problem Environment for Tufte-Book
% Add these packages to your preamble
% \usepackage{tcolorbox}
% \usepackage{xparse}
% \usepackage{enumitem}
% \usepackage{amsmath}
% \usepackage{listings}

% Configure tcolorbox for tufte compatibility
\tcbuselibrary{breakable,skins}

% Define colors that work well with tufte
\definecolor{problemtitle}{RGB}{107, 36, 31}
\definecolor{sectioncolor}{RGB}{107, 36, 31}
\definecolor{exampleback}{RGB}{248, 248, 248}
\definecolor{exampleborder}{RGB}{200, 200, 200}

% Configure listings for code blocks
\lstset{
    basicstyle=\ttfamily\small,
    breaklines=true,
    breakatwhitespace=true,
    frame=none,
    numbers=none,
    showstringspaces=false,
    tabsize=4,
    captionpos=b
}

% In your preamble
\newcounter{programmingproblem}[chapter] % or [section] if you want per-section numbering
\renewcommand{\theprogrammingproblem}{\arabic{programmingproblem}} % or \thesection.\arabic{programmingproblem} for section-based
% \newcommand{\listofproblems}{\listof{problems}{List of Problems}}
% \newcommand{\problemsname}{List of Problems}
\makeatletter
\newcommand{\listofproblems}{\section*{\problemsname}\@starttoc{problems}}
\makeatother
\newcommand{\problemsname}{List of Problems}

% Main programming problem environment
\NewDocumentEnvironment{programmingproblem}{m m m m m}{%
    \refstepcounter{programmingproblem}
    \begin{fullwidth}
    % Arguments: {title}{time_limit}{memory_limit}{problem_id}{label}
    \needspace{4\baselineskip}
    \vspace{\baselineskip}
    
    % Add to list of problems
    \addcontentsline{problems}{section}{\protect\numberline{\theprogrammingproblem}#1}
    
    % Problem title and metadata (left-aligned)
    \noindent{\Large\color{problemtitle}\textbf{Problem~\theprogrammingproblem. #1}}
    \vspace{0.25\baselineskip}
    
    \noindent{\small
    \textit{time limit per test:} #2 \\
    \textit{memory limit per test:} #3
    }
    \label{#5}

    \vspace{0.5\baselineskip}
}{%
    \vspace{\baselineskip}
    \end{fullwidth}
}

% Problem description command
\NewDocumentCommand{\problemdescription}{m}{%
    \noindent\raggedright #1
    \vspace{0.5\baselineskip}
}

% Input section
\NewDocumentCommand{\probleminput}{m}{%
    \vspace{0.5\baselineskip}
    \noindent{\color{sectioncolor}\textbf{Input}}
    
    \noindent #1
    \vspace{0.5\baselineskip}
}

% Output section  
\NewDocumentCommand{\problemoutput}{m}{%
    \vspace{0.5\baselineskip}
    \noindent{\color{sectioncolor}\textbf{Output}}
    
    \noindent #1
    \vspace{0.5\baselineskip}
}

% Examples environment
\NewDocumentEnvironment{problemexamples}{}{%
    \vspace{0.5\baselineskip}
    \noindent{\color{sectioncolor}\textbf{Examples}}
    \vspace{0.25\baselineskip}
}{%
    \vspace{0.5\baselineskip}
}

% Individual example environment
\NewDocumentEnvironment{example}{m m}{%
    % Arguments: {input_text}{output_text}
    \begin{tcolorbox}[
        colback=exampleback,
        colframe=exampleborder,
        boxrule=0.5pt,
        arc=2pt,
        left=5pt,
        right=5pt,
        top=5pt,
        bottom=5pt,
        breakable,
        before skip=0.5\baselineskip,
        after skip=0.5\baselineskip
    ]
    \begin{minipage}[t]{0.48\textwidth}
        \textbf{input}
        \vspace{2pt}
        
        {\ttfamily\small #1}
    \end{minipage}%
    \hfill%
    \begin{minipage}[t]{0.48\textwidth}
        \textbf{output}
        \vspace{2pt}
        
        {\ttfamily\small #2}
    \end{minipage}
    \end{tcolorbox}
}{}

% Alternative single-column example for very long inputs/outputs
\NewDocumentEnvironment{longexample}{m m}{%
    % Arguments: {input_text}{output_text}
    \begin{tcolorbox}[
        colback=exampleback,
        colframe=exampleborder,
        boxrule=0.5pt,
        arc=2pt,
        left=5pt,
        right=5pt,
        top=5pt,
        bottom=5pt,
        breakable,
        before skip=0.5\baselineskip,
        after skip=0.5\baselineskip
    ]
    \textbf{input}
    \vspace{2pt}
    
    {\ttfamily\small #1}
    
    \vspace{8pt}
    \textbf{output}
    \vspace{2pt}
    
    {\ttfamily\small #2}
    \end{tcolorbox}
}{}

% Notes section
\NewDocumentCommand{\problemnotes}{m}{%
    \vspace{0.5\baselineskip}
    \noindent{\color{sectioncolor}\textbf{Note}}
    
    \noindent #1
}

% Helper command for mathematical notation in problems
\NewDocumentCommand{\mathvar}{m}{\ensuremath{#1}}

% Helper command for sequence notation
\NewDocumentCommand{\sequence}{m}{\ensuremath{#1}}

% Helper command for highlighting important terms
\NewDocumentCommand{\term}{m}{\textit{#1}}

% Usage example:
% \begin{programmingproblem}{Sereja and Brackets}{1 second}{256 megabytes}{C}
%     \problemdescription{
%         Sereja has a bracket sequence \sequence{s_1, s_2, \ldots, s_n}, or, in other words, 
%         a string \mathvar{s} of length \mathvar{n}, consisting of characters ``('' and ``)''.
%         
%         Sereja needs to answer \mathvar{m} queries, each of them is described by two integers 
%         \mathvar{l_i, r_i} (\mathvar{1 \leq l_i \leq r_i \leq n}). The answer to the \mathvar{i}-th 
%         query is the length of the maximum \term{correct bracket subsequence} of sequence 
%         \sequence{s_{l_i}, s_{l_i + 1}, \ldots, s_{r_i}}. Help Sereja answer all queries.
%         
%         You can find the definitions for a \term{subsequence} and a \term{correct bracket sequence} 
%         in the notes.
%     }
%     
%     \probleminput{
%         The first line contains a sequence of characters \sequence{s_1, s_2, \ldots, s_n} 
%         (\mathvar{1 \leq n \leq 10^6}) without any spaces. Each character is either a ``('' or a ``)''. 
%         The second line contains integer \mathvar{m} (\mathvar{1 \leq m \leq 10^5}) --- the number of queries. 
%         Each of the next \mathvar{m} lines contains a pair of integers. The \mathvar{i}-th line contains 
%         integers \mathvar{l_i, r_i} (\mathvar{1 \leq l_i \leq r_i \leq n}) --- the description of the 
%         \mathvar{i}-th query.
%     }
%     
%     \problemoutput{
%         Print the answer to each question on a single line. Print the answers in the order they go in the input.
%     }
%     
%     \begin{problemexamples}
%         \begin{example}{
%             ())(())(())(\\
%             7\\
%             1 1\\
%             2 3\\
%             1 2\\
%             1 12\\
%             8 12\\
%             5 11\\
%             2 10
%         }{
%             0\\
%             0\\
%             2\\
%             10\\
%             4\\
%             6\\
%             6
%         }
%         \end{example}
%     \end{problemexamples}
%     
%     % For very long examples, use:
%     % \begin{longexample}{very long input here}{very long output here}
%     % \end{longexample}
%     
%     \problemnotes{
%         A \term{subsequence} of length \mathvar{|x|} of string \mathvar{s = s_1s_2\ldots s_{|s|}} 
%         (where \mathvar{|s|} is the length of string \mathvar{s}) is string 
%         \mathvar{x = s_{k_1}s_{k_2}\ldots s_{k_{|x|}}} (\mathvar{1 \leq k_1 < k_2 < \ldots < k_{|x|} \leq |s|}).
%         
%         A \term{correct bracket sequence} is a bracket sequence that can be transformed into a correct 
%         arithmetic expression by inserting characters ``1'' and ``+'' between the characters of the string. 
%         For example, bracket sequences ``()()'', ``(())'', are correct (the resulting expressions 
%         ``(1)+(1)'', ``((1+1)+1)''), and ``)('' and ``('' are not.
%         
%         For the third query required sequence will be \sequence{()}.
%         
%         For the fourth query required sequence will be \sequence{()()()()}.
%     }
% \end{programmingproblem}