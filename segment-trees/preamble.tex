\usepackage{hyperref}
\hypersetup{colorlinks}% uncomment this line if you prefer colored hyperlinks (e.g., for onscreen viewing)

% =========================
% Typography and Spacing
% =========================
\usepackage{microtype} % Improves character and word spacing
\usepackage{xspace}    % Better handling of trailing spaces in macros
\usepackage{units}     % For typesetting units
\usepackage{xparse}
\usepackage[utf8]{inputenc}
\usepackage[T1]{fontenc} 
\usepackage{enumitem}
\usepackage{needspace}
\usepackage[most]{tcolorbox}
\usepackage{tocloft}
\usepackage{caption}


% =========================
% Math
% =========================
\usepackage{amsmath}

% Generates the index
\usepackage{makeidx}

%%
% If they're installed, use Bergamo and Chantilly from www.fontsite.com.
% They're clones of Bembo and Gill Sans, respectively.
\IfFileExists{bergamo.sty}{\usepackage[osf]{bergamo}}{}% Bembo
\IfFileExists{chantill.sty}{\usepackage{chantill}}{}% Gill Sans


\usepackage{import}
\usepackage{listings}
\usepackage[vlined]{algorithm2e}

%%
% Just some sample text
\usepackage{lipsum}

%%
% For nicely typeset tabular material
\usepackage{booktabs}

%%
% For graphics / images
\usepackage{tikz}
\usetikzlibrary{fit,positioning,calc,arrows.meta}
\usepackage{graphicx}
\setkeys{Gin}{width=\linewidth,totalheight=\textheight,keepaspectratio}
\graphicspath{{graphics/}}

% The fancyvrb package lets us customize the formatting of verbatim
% environments.  We use a slightly smaller font.
\usepackage{fancyvrb}
\fvset{fontsize=\normalsize}


% --- Math and Complexity ---
\newcommand{\bigO}[1]{\ensuremath{\mathcal{O}(#1)}} % Big O notation

%%%%%%%%%%%%%%%%%%%%%%%%%%%%%%%%%%%%%%%%%%%%%%%%%%%%%%%%%%%%%%%%%%%%%%%%%%%%%%%
% Marginlisting and marginnoteenv
%%%%%%%%%%%%%%%%%%%%%%%%%%%%%%%%%%%%%%%%%%%%%%%%%%%%%%%%%%%%%%%%%%%%%%%%%%%%%%%
\newsavebox{\marginlistingbox}
\newsavebox{\marginnotecontentbox}

% --- marginlisting Environment ---
% For putting code listings in the margin
\newenvironment{marginlisting}[2][0pt]%
% #1: Optional argument for vertical space BEFORE the minipage in the margin (default 0pt)
% #2: Mandatory argument for \lstset options (e.g., "language=C++,caption=My Code")
  {\def\marginlistingvspace{#1}%  % Save #1 in a macro
   \begin{lrbox}{\marginlistingbox}%
   \vspace*{\marginlistingvspace}%
   \begin{minipage}{\marginparwidth}%
     \setlength{\parindent}{0pt}%       Set paragraph indent to 0 for any text in this minipage
     \setlength{\leftskip}{0pt}%        Set left skip to 0 for any text in this minipage
     % \RaggedRight % Uncomment if you want general text (not listings) in the margin to be ragged right
     \footnotesize % Default font size for general text within this minipage
                   % (lstlisting's basicstyle can override for code)
     \lstset{ % Settings specifically for lstlisting environments *within* this marginlisting environment
       xleftmargin=0pt,         % --- KEY: Remove listings' own left margin relative to its container (the minipage) ---
       aboveskip=1pt,           % Minimal vertical space above the listing block
       belowskip=1pt,           % Minimal vertical space below the listing block
       basicstyle=\ttfamily\scriptsize, % Default style for code in margin listings (can be overridden by #2)
       breaklines=true,         % Allow automatic line breaking for long lines
       breakatwhitespace=true,  % Prefer to break lines at white space
       keepspaces=true,         % Preserve spaces in the code, important for formatting
       showstringspaces=false,  % Do not show special symbols for spaces in strings
       captionpos=b,            % Caption position (bottom)
       #2                     % Apply user-provided lstlisting options passed as the second argument
     }%
   % The content of the \begin{marginlisting}...\end{marginlisting} goes here,
   % which should typically include a \begin{lstlisting}...\end{lstlisting} block
  }
  {\end{minipage}\end{lrbox}\marginpar{\vspace{\marginlistingvspace}\usebox{\marginlistingbox}}} % End minipage, end lrbox, and place it in the margin with vertical adjustment

% --- marginnoteenv Environment ---
% For putting general text notes in the margin
\newenvironment{marginnoteenv}[1][0pt]%
% #1: Optional argument for vertical space BEFORE the minipage in the margin (default 0pt)
  {\def\marginnotevspace{#1}%
   \begin{lrbox}{\marginnotecontentbox}%
   \begin{minipage}{\marginparwidth}%
     \setlength{\parindent}{0pt}%
     \setlength{\leftskip}{0pt}%
     \footnotesize
  }
  {\end{minipage}\end{lrbox}%
   \marginpar{\vspace{\marginnotevspace}\usebox{\marginnotecontentbox}}}

  

%%
% Prints argument within hanging parentheses (i.e., parentheses that take
% up no horizontal space).  Useful in tabular environments.
\newcommand{\hangp}[1]{\makebox[0pt][r]{(}#1\makebox[0pt][l]{)}}

%%
% Prints an asterisk that takes up no horizontal space.
% Useful in tabular environments.
\newcommand{\hangstar}{\makebox[0pt][l]{*}}

%%

%%
% Some shortcuts for Tufte's book titles.  The lowercase commands will
% produce the initials of the book title in italics.  The all-caps commands
% will print out the full title of the book in italics.
\newcommand{\vdqi}{\textit{VDQI}\xspace}
\newcommand{\ei}{\textit{EI}\xspace}
\newcommand{\ve}{\textit{VE}\xspace}
\newcommand{\be}{\textit{BE}\xspace}
\newcommand{\VDQI}{\textit{The Visual Display of Quantitative Information}\xspace}
\newcommand{\EI}{\textit{Envisioning Information}\xspace}
\newcommand{\VE}{\textit{Visual Explanations}\xspace}
\newcommand{\BE}{\textit{Beautiful Evidence}\xspace}

\newcommand{\TL}{Tufte-\LaTeX\xspace}

% Prints the month name (e.g., January) and the year (e.g., 2008)
\newcommand{\monthyear}{%
  \ifcase\month\or January\or February\or March\or April\or May\or June\or
  July\or August\or September\or October\or November\or
  December\fi\space\number\year
}


% Prints an epigraph and speaker in sans serif, all-caps type.
\newcommand{\openepigraph}[2]{%
  %\sffamily\fontsize{14}{16}\selectfont
  \begin{fullwidth}
  \sffamily\large
  \begin{doublespace}
  \noindent\allcaps{#1}\\% epigraph
  \noindent\allcaps{#2}% author
  \end{doublespace}
  \end{fullwidth}
}

% Inserts a blank page
\newcommand{\blankpage}{\newpage\hbox{}\thispagestyle{empty}\newpage}


% Typesets the font size, leading, and measure in the form of 10/12x26 pc.
\newcommand{\measure}[3]{#1/#2$\times$\unit[#3]{pc}}

% Macros for typesetting the documentation
\newcommand{\hlred}[1]{\textcolor{Maroon}{#1}}% prints in red
\newcommand{\hangleft}[1]{\makebox[0pt][r]{#1}}
\newcommand{\hairsp}{\hspace{1pt}}% hair space
\newcommand{\hquad}{\hskip0.5em\relax}% half quad space
\newcommand{\TODO}{\textcolor{red}{\bf TODO!}\xspace}
\newcommand{\na}{\quad--}% used in tables for N/A cells
\providecommand{\XeLaTeX}{X\lower.5ex\hbox{\kern-0.15em\reflectbox{E}}\kern-0.1em\LaTeX}
\newcommand{\tXeLaTeX}{\XeLaTeX\index{XeLaTeX@\protect\XeLaTeX}}
% \index{\texttt{\textbackslash xyz}@\hangleft{\texttt{\textbackslash}}\texttt{xyz}}
\newcommand{\tuftebs}{\symbol{'134}}% a backslash in tt type in OT1/T1
\newcommand{\doccmdnoindex}[2][]{
  \texttt{\tuftebs#2}%
  \ifthenelse{\isempty{#1}}%
    {% add the command to the index
      \index{#2 command@\protect\hangleft{\texttt{\tuftebs}}\texttt{#2}}% command name
    }%
    {% add the command and package to the index
      \index{#2 command@\protect\hangleft{\texttt{\tuftebs}}\texttt{#2} (\texttt{#1} package)}% command name
      \index{#1 package@\texttt{#1} package}\index{packages!#1@\texttt{#1}}% package name
    }%
}% command name -- adds backslash automatically
\newcommand{\doccmddef}[2][]{%
  \hlred{\texttt{\tuftebs#2}}\label{cmd:#2}%
  \ifthenelse{\isempty{#1}}%
    {% add the command to the index
      \index{#2 command@\protect\hangleft{\texttt{\tuftebs}}\texttt{#2}}% command name
    }%
    {% add the command and package to the index
      \index{#2 command@\protect\hangleft{\texttt{\tuftebs}}\texttt{#2} (\texttt{#1} package)}% command name
      \index{#1 package@\texttt{#1} package}\index{packages!#1@\texttt{#1}}% package name
    }%
}% command name -- adds backslash automatically
\newcommand{\doccmd}[2][]{%
  \texttt{\tuftebs#2}%
  \ifthenelse{\isempty{#1}}%
    {% add the command to the index
      \index{#2 command@\protect\hangleft{\texttt{\tuftebs}}\texttt{#2}}% command name
    }%
    {% add the command and package to the index
      \index{#2 command@\protect\hangleft{\texttt{\tuftebs}}\texttt{#2} (\texttt{#1} package)}% command name
      \index{#1 package@\texttt{#1} package}\index{packages!#1@\texttt{#1}}% package name
    }%
}% command name -- adds backslash automatically
\newcommand{\docopt}[1]{\ensuremath{\langle}\textrm{\textit{#1}}\ensuremath{\rangle}}% optional command argument
\newcommand{\docarg}[1]{\textrm{\textit{#1}}}% (required) command argument
\newenvironment{docspec}{\begin{quotation}\ttfamily\parskip0pt\parindent0pt\ignorespaces}{\end{quotation}}% command specification environment
\newcommand{\docenv}[1]{\texttt{#1}\index{#1 environment@\texttt{#1} environment}\index{environments!#1@\texttt{#1}}}% environment name
\newcommand{\docenvdef}[1]{\hlred{\texttt{#1}}\label{env:#1}\index{#1 environment@\texttt{#1} environment}\index{environments!#1@\texttt{#1}}}% environment name
\newcommand{\docpkg}[1]{\texttt{#1}\index{#1 package@\texttt{#1} package}\index{packages!#1@\texttt{#1}}}% package name
\newcommand{\doccls}[1]{\texttt{#1}}% document class name
\newcommand{\docclsopt}[1]{\texttt{#1}\index{#1 class option@\texttt{#1} class option}\index{class options!#1@\texttt{#1}}}% document class option name
\newcommand{\docclsoptdef}[1]{\hlred{\texttt{#1}}\label{clsopt:#1}\index{#1 class option@\texttt{#1} class option}\index{class options!#1@\texttt{#1}}}% document class option name defined
\newcommand{\docmsg}[2]{\bigskip\begin{fullwidth}\noindent\ttfamily#1\end{fullwidth}\medskip\par\noindent#2}
\newcommand{\docfilehook}[2]{\texttt{#1}\index{file hooks!#2}\index{#1@\texttt{#1}}}
\newcommand{\doccounter}[1]{\texttt{#1}\index{#1 counter@\texttt{#1} counter}}



\newtcolorbox{intuition}{
  colback=gray!10,
  colframe=gray!50,
  boxrule=0.5pt,
  arc=0.5mm,
  left=1mm, right=1mm, top=1mm, bottom=1mm,
  fontupper=\small,
  before skip=10pt, after skip=10pt
}
  
\lstset{
  basicstyle   = \footnotesize\ttfamily,
  columns      = fullflexible,
  resetmargins = true,     % forget any list indent
  xleftmargin  = 0pt,      % really flush with margin
  numbers      = none,
  frame        = none,
  aboveskip    = 0pt,
  belowskip    = 0pt
}

\lstdefinestyle{mypython}{
    language=Python,
    basicstyle=\ttfamily\small,
    keywordstyle=\color{blue},
    stringstyle=\color{red},
    commentstyle=\color{gray},
    numbers=none,
    numberstyle=\tiny\color{gray},
    stepnumber=1,
    numbersep=5pt,
    showstringspaces=false,
    tabsize=4,
    breaklines=true,
    frame=none,
    backgroundcolor=\color{white}
}

\lstset{style=mypython}

% Programming Problem Environment for Tufte-Book
% Add these packages to your preamble
% \usepackage{tcolorbox}
% \usepackage{xparse}
% \usepackage{enumitem}
% \usepackage{amsmath}
% \usepackage{listings}

% Configure tcolorbox for tufte compatibility
\tcbuselibrary{breakable,skins}

% Define colors that work well with tufte
\definecolor{problemtitle}{RGB}{107, 36, 31}
\definecolor{sectioncolor}{RGB}{107, 36, 31}
\definecolor{exampleback}{RGB}{248, 248, 248}
\definecolor{exampleborder}{RGB}{200, 200, 200}

% Configure listings for code blocks
\lstset{
    basicstyle=\ttfamily\small,
    breaklines=true,
    breakatwhitespace=true,
    frame=none,
    numbers=none,
    showstringspaces=false,
    tabsize=4,
    captionpos=b
}

% In your preamble
\newcounter{programmingproblem}[chapter] % or [section] if you want per-section numbering
\renewcommand{\theprogrammingproblem}{\arabic{programmingproblem}} % or \thesection.\arabic{programmingproblem} for section-based
% \newcommand{\listofproblems}{\listof{problems}{List of Problems}}
% \newcommand{\problemsname}{List of Problems}
\makeatletter
\newcommand{\listofproblems}{\section*{\problemsname}\@starttoc{problems}}
\makeatother
\newcommand{\problemsname}{List of Problems}

% Main programming problem environment
\NewDocumentEnvironment{programmingproblem}{m m m m m}{%
    \refstepcounter{programmingproblem}
    \begin{fullwidth}
    % Arguments: {title}{time_limit}{memory_limit}{problem_id}{label}
    \needspace{4\baselineskip}
    \vspace{\baselineskip}
    
    % Add to list of problems
    \addcontentsline{problems}{section}{\protect\numberline{\theprogrammingproblem}#1}
    
    % Problem title and metadata (left-aligned)
    \noindent{\Large\color{problemtitle}\textbf{Problem~\theprogrammingproblem. #1}}
    \vspace{0.25\baselineskip}
    
    \noindent{\small
    \textit{time limit per test:} #2 \\
    \textit{memory limit per test:} #3
    }
    \label{#5}

    \vspace{0.5\baselineskip}
}{%
    \vspace{\baselineskip}
    \end{fullwidth}
}

% Problem description command
\NewDocumentCommand{\problemdescription}{m}{%
    \noindent\raggedright #1
    \vspace{0.5\baselineskip}
}

% Input section
\NewDocumentCommand{\probleminput}{m}{%
    \vspace{0.5\baselineskip}
    \noindent{\color{sectioncolor}\textbf{Input}}
    
    \noindent #1
    \vspace{0.5\baselineskip}
}

% Output section  
\NewDocumentCommand{\problemoutput}{m}{%
    \vspace{0.5\baselineskip}
    \noindent{\color{sectioncolor}\textbf{Output}}
    
    \noindent #1
    \vspace{0.5\baselineskip}
}

% Examples environment
\NewDocumentEnvironment{problemexamples}{}{%
    \vspace{0.5\baselineskip}
    \noindent{\color{sectioncolor}\textbf{Examples}}
    \vspace{0.25\baselineskip}
}{%
    \vspace{0.5\baselineskip}
}

% Individual example environment
\NewDocumentEnvironment{example}{m m}{%
    % Arguments: {input_text}{output_text}
    \begin{tcolorbox}[
        colback=exampleback,
        colframe=exampleborder,
        boxrule=0.5pt,
        arc=2pt,
        left=5pt,
        right=5pt,
        top=5pt,
        bottom=5pt,
        breakable,
        before skip=0.5\baselineskip,
        after skip=0.5\baselineskip
    ]
    \begin{minipage}[t]{0.48\textwidth}
        \textbf{input}
        \vspace{2pt}
        
        {\ttfamily\small #1}
    \end{minipage}%
    \hfill%
    \begin{minipage}[t]{0.48\textwidth}
        \textbf{output}
        \vspace{2pt}
        
        {\ttfamily\small #2}
    \end{minipage}
    \end{tcolorbox}
}{}

% Alternative single-column example for very long inputs/outputs
\NewDocumentEnvironment{longexample}{m m}{%
    % Arguments: {input_text}{output_text}
    \begin{tcolorbox}[
        colback=exampleback,
        colframe=exampleborder,
        boxrule=0.5pt,
        arc=2pt,
        left=5pt,
        right=5pt,
        top=5pt,
        bottom=5pt,
        breakable,
        before skip=0.5\baselineskip,
        after skip=0.5\baselineskip
    ]
    \textbf{input}
    \vspace{2pt}
    
    {\ttfamily\small #1}
    
    \vspace{8pt}
    \textbf{output}
    \vspace{2pt}
    
    {\ttfamily\small #2}
    \end{tcolorbox}
}{}

% Notes section
\NewDocumentCommand{\problemnotes}{m}{%
    \vspace{0.5\baselineskip}
    \noindent{\color{sectioncolor}\textbf{Note}}
    
    \noindent #1
}

% Helper command for mathematical notation in problems
\NewDocumentCommand{\mathvar}{m}{\ensuremath{#1}}

% Helper command for sequence notation
\NewDocumentCommand{\sequence}{m}{\ensuremath{#1}}

% Helper command for highlighting important terms
\NewDocumentCommand{\term}{m}{\textit{#1}}

% Usage example:
% \begin{programmingproblem}{Sereja and Brackets}{1 second}{256 megabytes}{C}
%     \problemdescription{
%         Sereja has a bracket sequence \sequence{s_1, s_2, \ldots, s_n}, or, in other words, 
%         a string \mathvar{s} of length \mathvar{n}, consisting of characters ``('' and ``)''.
%         
%         Sereja needs to answer \mathvar{m} queries, each of them is described by two integers 
%         \mathvar{l_i, r_i} (\mathvar{1 \leq l_i \leq r_i \leq n}). The answer to the \mathvar{i}-th 
%         query is the length of the maximum \term{correct bracket subsequence} of sequence 
%         \sequence{s_{l_i}, s_{l_i + 1}, \ldots, s_{r_i}}. Help Sereja answer all queries.
%         
%         You can find the definitions for a \term{subsequence} and a \term{correct bracket sequence} 
%         in the notes.
%     }
%     
%     \probleminput{
%         The first line contains a sequence of characters \sequence{s_1, s_2, \ldots, s_n} 
%         (\mathvar{1 \leq n \leq 10^6}) without any spaces. Each character is either a ``('' or a ``)''. 
%         The second line contains integer \mathvar{m} (\mathvar{1 \leq m \leq 10^5}) --- the number of queries. 
%         Each of the next \mathvar{m} lines contains a pair of integers. The \mathvar{i}-th line contains 
%         integers \mathvar{l_i, r_i} (\mathvar{1 \leq l_i \leq r_i \leq n}) --- the description of the 
%         \mathvar{i}-th query.
%     }
%     
%     \problemoutput{
%         Print the answer to each question on a single line. Print the answers in the order they go in the input.
%     }
%     
%     \begin{problemexamples}
%         \begin{example}{
%             ())(())(())(\\
%             7\\
%             1 1\\
%             2 3\\
%             1 2\\
%             1 12\\
%             8 12\\
%             5 11\\
%             2 10
%         }{
%             0\\
%             0\\
%             2\\
%             10\\
%             4\\
%             6\\
%             6
%         }
%         \end{example}
%     \end{problemexamples}
%     
%     % For very long examples, use:
%     % \begin{longexample}{very long input here}{very long output here}
%     % \end{longexample}
%     
%     \problemnotes{
%         A \term{subsequence} of length \mathvar{|x|} of string \mathvar{s = s_1s_2\ldots s_{|s|}} 
%         (where \mathvar{|s|} is the length of string \mathvar{s}) is string 
%         \mathvar{x = s_{k_1}s_{k_2}\ldots s_{k_{|x|}}} (\mathvar{1 \leq k_1 < k_2 < \ldots < k_{|x|} \leq |s|}).
%         
%         A \term{correct bracket sequence} is a bracket sequence that can be transformed into a correct 
%         arithmetic expression by inserting characters ``1'' and ``+'' between the characters of the string. 
%         For example, bracket sequences ``()()'', ``(())'', are correct (the resulting expressions 
%         ``(1)+(1)'', ``((1+1)+1)''), and ``)('' and ``('' are not.
%         
%         For the third query required sequence will be \sequence{()}.
%         
%         For the fourth query required sequence will be \sequence{()()()()}.
%     }
% \end{programmingproblem}

% In your preamble, after \usepackage{tocloft} if you use it

% Define how each entry in the list looks
\makeatletter
\newcommand{\l@problems}[2]{\@dottedtocline{1}{1.5em}{2.3em}{#1}{#2}}
\makeatother

\DeclareCaptionFont{scriptsize}{\scriptsize}
\captionsetup[lstlisting]{font=scriptsize}

% Custom command for problem marginnote with pill style
\newcommand{\problemrefmarginnote}[3][0pt]{%
  \begin{marginnoteenv}[#1]
    \tcbox[colback=gray!15, colframe=gray!50, boxrule=0pt, arc=4pt, left=4pt, right=4pt, top=2pt, bottom=2pt, boxsep=0pt, enhanced]{%
      Problem~\ref{#2} (#3)
    }
  \end{marginnoteenv}
}


